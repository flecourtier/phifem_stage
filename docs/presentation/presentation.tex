%% Requires compilation with XeLaTeX or LuaLaTeX
\documentclass[compress,10pt,xcolor={table,dvipsnames},t]{beamer}
%\documentclass[compress]{beamer}
\usetheme{diapo}
\usepackage{amsmath}
\usepackage[bottom]{footmisc}
\usepackage{multirow}
\usepackage{setspace}
\usepackage{caption}
\usepackage{array,multirow,makecell}
\usepackage[table]{xcolor}
\usepackage{pifont}
\usepackage{hyperref}
\usepackage[utf8]{inputenc}
\setcellgapes{1pt}
\setlength{\parindent}{0pt}
\makegapedcells
\newcolumntype{R}[1]{>{\raggedleft\arraybackslash }b{#1}}
\newcolumntype{L}[1]{>{\raggedright\arraybackslash }b{#1}}
\newcolumntype{C}[1]{>{\centering\arraybackslash }b{#1}}
\usepackage{paralist}

\usepackage[backend=biber,style=numeric,sorting=nyt]{biblatex}
\renewcommand*{\bibfont}{\scriptsize}
\addbibresource{biblio.bib}

\hypersetup{
	colorlinks=true,
	urlcolor=blue,
	citecolor=blue,
	linkcolor=titre,
}

\title[PhiFEM]{Innovative non-conformal finite element methods for augmented surgery}
\subtitle{Internship presentation}
\author[name]{LECOURTIER Frédérique, DUPREZ Michel, FRANCK Emmanuel, LLERAS Vanessa}
\institute{\large Strasbourg University}
\date{\today}

\useoutertheme[subsection=false]{miniframes}
\usepackage{etoolbox}
\makeatletter
\patchcmd{\slideentry}{\advance\beamer@xpos by1\relax}{}{}{}
\def\beamer@subsectionentry#1#2#3#4#5{\advance\beamer@xpos by1\relax}%
\makeatother

\allowbreak

\begin{document}
	\nocite{*}
	
	\begin{frame}
		\vspace{-20pt}
		\titlepage
	\end{frame}
	
	\AtBeginSection[]{
		\begin{frame}
			\vfill
			\centering
			\begin{beamercolorbox}[sep=5pt,shadow=true,rounded=true]{subtitle}
				\usebeamerfont{title}\insertsectionhead\par%
			\end{beamercolorbox}
			%\tableofcontents[sectionstyle=hide,subsectionstyle=show]
			\tableofcontents[sectionstyle=hide,subsectionstyle=show/shaded/hide]
			\vfill
		\end{frame}
	}

	\AtBeginSubsection[]{
		\begin{frame}
			\vfill
			\centering
			\begin{beamercolorbox}[sep=5pt,shadow=true,rounded=true]{subtitle}
				\usebeamerfont{title}\insertsectionhead\par%
			\end{beamercolorbox}
			\tableofcontents[sectionstyle=hide,subsectionstyle=show/shaded/hide]
			\vfill
		\end{frame}
	}

	\section{Introduction}

	\begin{frame}{Presentation of the teams}
		\begin{center}
			\pgfimage[width=0.3\linewidth]{images/intro/logo-mimesis.png}
		\end{center}
		\begin{enumerate}[\ding{217}]
			\item project-team as sub-team of MLMS ("Machine Leraning, Modélisation et Simulation") of Inria
			\item \textbf{Aim :} to create real-time digital twins of an organ
			\item \textbf{Scientific challenges :}
			\begin{itemize}
				\item scientific computing
				\item data assimilation
				\item machine learning
				\item control
			\end{itemize}
			\item \textbf{Main application domains :}
			\begin{itemize}
				\item surgical training
				\item surgical guidance during complex interventions
			\end{itemize}
		\end{enumerate}
	\end{frame}

	\begin{frame}{Scientific context}
		\color{red}A COMPLETER !!\color{black}
%		Because of the geometry of organs, mimesis has developed a new method: the $\phi$-FEM method. \\
%		\textbf{Abstract :} Mimesis propose a new fictitious domain finite element method, well suited for elliptic problems posed
%		in a domain given by a level-set function without requiring a mesh fitting the boundary. 
%		\begin{center}
%			\pgfimage[width=0.8\linewidth]{images/intro/context_geometry.jpg}
%		\end{center}
	\end{frame}

	\begin{frame}{Objectives - Deliverables}
		\underline{\textbf{Objectives :}} \\
		\begin{enumerate}[\ding{217}]
			\item Train a Fourier Neural Operator (FNO), with $\phi$-FEM solutions, to predict the solutions of a given PDE.
			\item Apply a correction on the FNO predictions.
		\end{enumerate}
		\textbf{Aim :} Neural networks are very fast, but not very accurate \\
		=> finite element methods are used to improve prediction accuracy.
		
		\underline{\textbf{Deliverables :}}

		\begin{enumerate}[\ding{217}]
			\item a \href{https://github.com/flecourtier/phifem_stage/blob/main/docs/suivi/suivi.pdf}{weekly tracking report} ( in French)
			\item a \href{https://github.com/flecourtier/phifem_stage}{github repository} containing all the code allowing to reproduce the results presented in this report
			\item a \href{https://csmi.cemosis.fr/csmi-stages-2023/m2/_attachments/Lecourtier-Fr\%C3\%A9d\%C3\%A9rique.pdf}{report} of the internship
			\item an \href{https://flecourtier.github.io/phifem_stage/phifem_project/1.0.3/main_page.html}{online report} generated with a tool called antora (made by a github CI)
		\end{enumerate}
	\end{frame}

	\begin{frame}{Problem considered}
		\textbf{Poisson problem with Dirichlet conditions :} \\
		Find $u : \Omega \rightarrow \mathbb{R}^d (d=1,2,3)$ such that
		\begin{equation*}
			\left\{
			\begin{aligned}
				-\Delta u &= f, \; &&\text{in } \; \Omega, \\
				u&=g, \; &&\text{on } \; \partial\Omega,
			\end{aligned}
			\right.
		\end{equation*}
		with $\Delta$ the Laplace operator, $\Omega$ a smooth bounded open set and $\partial\Omega$ its boundary.
	\end{frame}

	\section{General methods and tools}
	
	\subsection{Standard FEM method}
	
	\begin{frame}{Presentation of standard FEM method}	
		\textbf{Variational Problem :} 
		\begin{equation*}
			\text{Find } u\in V \text{ such that } a(u,v)=l(v), \;\forall v\in V
		\end{equation*}
		where $V$ is a Hilbert space, $a$ is a bilinear form and $l$ is a linear form.
		
		\textbf{Approach Problem :} 
		\begin{equation*}
			\text{Find } u_h\in V_h \text{ such that } a(u_h,v_h)=l(v_h), \;\forall v_h\in V
		\end{equation*}
		with $u_h$ an approximate solution in $V_h$, a finite-dimensional space dependent on $h$ such that $\quad V_h\subset V, \; dimV_h = N_h<\infty, \; \forall h>0$ 
		
		As $u_h=\sum_{i=1}^{N_h}u_i\varphi_i$ with $(\varphi_1,\dots,\varphi_{N_h})$ a basis of $V_h$, finding an approximation of the PDE solution implies solving the following linear system:
		\begin{equation*}
			AU=b
		\end{equation*}
		with
		\begin{equation*}
			A=(a(\varphi_i,\varphi_j))_{1\le i,j\le N_h}, \quad U=(u_i)_{1\le i\le N_h} \quad \text{and} \quad b=(l(\varphi_j))_{1\le j\le N_h}
		\end{equation*}
	\end{frame}

	\begin{frame}{In practice}
		\begin{enumerate}[\ding{217}]
			\item \begin{minipage}[t]{0.68\linewidth}
				Construct a mesh of our $\Omega$ geometry with a family of elements (in 2D: triangle, rectangle; in 3D: tetrahedron, parallelepiped, prism) defined by
				$$\mathcal{T}_h = \left\{K_1,\dots,K_{N_e}\right\}$$
				where $N_e$ is the number of elements. \\
			\end{minipage} \begin{minipage}[t][][b]{0.28\linewidth}
				\centering
				\qquad \pgfimage[width=0.8\linewidth]{images/methods_tools/FEM_triangle_mesh.png}
			\end{minipage}
			\item Construct a space of piece-wise affine continuous functions, defined by
			\begin{equation*}
				V_h:=P_{C,h}^k=\{v_h\in C^0(\bar{\Omega}), \forall K\in\mathcal{T}_h, {v_h}_{|K}\in\mathbb{P}_k\}
			\end{equation*}
			where $\mathbb{P}_k$ is a vector space of polynomials of total degree less than or equal to $k$.
		\end{enumerate}
		
	\end{frame}

	\subsection{$\phi$-FEM method}
	
	\begin{frame}{Context}
		\textbf{Idea :} $\phi$-FEM method = new fictitious domain finite element method that does not require a mesh conforming to the real boundary.
		\begin{center}
			\pgfimage[width=0.7\linewidth]{images/methods_tools/PhiFEM_context_mesh.png}
		\end{center}
		\textbf{Advantage :} boundary represented by a level-set function $\Rightarrow$ only this function will change over time during real-time simulation
	\end{frame}
	
	\begin{frame}{Problem}
		We pose $u=\phi w$ such that
		$$\left\{\begin{aligned}
			-\Delta (\phi w) &= f, \; \text{on } \Omega, \\
			u&=g, \; \text{in } \Gamma, \\
		\end{aligned}\right.$$
		where $\phi$ is the level-set function and $\Omega$ and $\Gamma$ are given by :
		$$
		\Omega=\{\phi < 0\} \quad \text{and} \quad \Gamma=\{\phi = 0\}.$$
		\begin{center}
			\pgfimage[width=0.3\linewidth]{images/methods_tools/PhiFEM_level_set.png}
		\end{center}
		The level-set function $\phi$ is supposed to be known on $\mathbb{R}^d$ and sufficiently smooth. \\
		For instance, the signed distance to $\Gamma$ is a good candidate
	\end{frame}
	
	\begin{frame}{Fictitious domain}
		\setstretch{0.5}
		\begin{minipage}{0.39\linewidth}
			\centering
			\pgfimage[width=\linewidth]{images/methods_tools/PhiFEM_domain.png} \\ \tiny \; \\
			\pgfimage[width=0.15\linewidth]{images/methods_tools/PhiFEM_fleche.png} \qquad \qquad \qquad \qquad  \\
			\tiny \pgfimage[width=\linewidth]{images/methods_tools/PhiFEM_domain_considered.png}
		\end{minipage}
		\begin{minipage}{0.6\linewidth}
			\begin{enumerate}[\ding{217}]
				\item $\mathcal{O}$ : fictitious domain such that $\Omega\subset\mathcal{O}$
				\item $\mathcal{T}_h^\mathcal{O}$ : simple quasi-uniform mesh on $\mathcal{O}$
				\item $\phi_h=I_{h,\mathcal{O}}^{(l)}(\phi)\in V_{h,\mathcal{O}}^{(l)}$ : approximation of $\phi$ \\ 
				with $I_{h,\mathcal{O}}^{(l)}$ the standard Lagrange interpolation operator on
				$$V_{h,\mathcal{O}}^{(l)}=\left\{v_h\in H^1(\mathcal{O}):v_{h|_T}\in\mathbb{P}_l(T) \;  \forall T\in\mathcal{T}_h^\mathcal{O}\right\}$$
				\item $\Gamma_h=\{\phi_h=0\}$ : approximate boundary of $\Gamma$
				\item $\mathcal{T}_h$ : sub-mesh of $\mathcal{T}_h^\mathcal{O}$ defined by
				$$\mathcal{T}_h=\left\{T\in \mathcal{T}_h^\mathcal{O}:T\cap\{\phi_h<0\}\ne\emptyset\right\}$$
				\item $\Omega_h$ : domain covered by the $\mathcal{T}_h$ mesh defined by
				$$\Omega_h=\left(\cup_{T\in\mathcal{T}_h}T\right)^O$$
				($\partial\Omega_h$ its boundary)
			\end{enumerate}			
		\end{minipage}
	\end{frame}
	
	\begin{frame}{Facets and Cells sets}
		\begin{minipage}{0.38\linewidth}
			\centering
			\pgfimage[width=\linewidth]{images/methods_tools/PhiFEM_boundary_cells.png} \\ \; \\ \; \\
			\pgfimage[width=\linewidth]{images/methods_tools/PhiFEM_boundary_edges.png}
		\end{minipage} \;
		\begin{minipage}{0.6\linewidth}
			\begin{enumerate}[\ding{217}]
				\item $\mathcal{T}_h^\Gamma\subset \mathcal{T}_h$ : contains the mesh elements cut by $\Gamma_h$, i.e. 
				\begin{equation*}
					\mathcal{T}_h^\Gamma=\left\{T\in\mathcal{T}_h:T\cap\Gamma_h\ne\emptyset\right\},
				\end{equation*}
				\item $\Omega_h^\Gamma$ : domain covered by the $\mathcal{T}_h^\Gamma$ mesh, i.e.
				\begin{equation*}
					\Omega_h^\Gamma=\left(\cup_{T\in\mathcal{T}_h^\Gamma}T\right)^O
				\end{equation*}
				\item $\mathcal{F}_h^\Gamma$ : collects the interior facets of $\mathcal{T}_h$ either cut by $\Gamma_h$ or belonging to a cut mesh element, i.e.
				\begin{align*}
					\mathcal{F}_h^\Gamma=\left\{E\;(\text{an internal facet of } \mathcal{T}_h) \text{ such that }\right. \\
					\left. \exists T\in \mathcal{T}_h:T\cap\Gamma_h\ne\emptyset \text{ and } E\in\partial T\right\}
				\end{align*}
			\end{enumerate}
		\end{minipage}
	\end{frame}

	\begin{frame}{Application to the Poisson problem}
		We start by consider the \textbf{homogeneous case} ($g=0$ on $\Gamma$). \\
		\textbf{Approach Problem :} Find $w_h\in V_h^{(k)}$ such that 
		$$a_h(w_h,v_h) = l_h(v_h) \quad \forall v_h \in V_h^{(k)}$$
		where
		$$a_h(w,v)=\int_{\Omega_h} \nabla (\phi_h w) \cdot \nabla (\phi_h v) - \int_{\partial\Omega_h} \frac{\partial}{\partial n}(\phi_h w)\phi_h v+G_h(w,v),$$
		$$l_h(v)=\int_{\Omega_h} f \phi_h v + G_h^{rhs}(v)$$
		and 
		$$V_h^{(k)}=\left\{v_h\in H^1(\Omega_h):v_{h|_T}\in\mathbb{P}_k(T), \; \forall T\in\mathcal{T}_h\right\}.$$
	\end{frame}

	\begin{frame}{Stabilization terms}
		\begin{center}
			\centering
			\pgfimage[width=\linewidth]{images/methods_tools/PhiFEM_stab_terms.png}
		\end{center}
		\small
		\underline{1st term :} Ghost penality \cite{burman_ghost_2010}, ensure continuity of the solution by penalizing gradient jumps. \\
		\underline{2nd term :} require the solution to verify the strong form on $\Omega_h^\Gamma$. \\
		\normalsize
		\textbf{Purpose :} 
		\begin{enumerate}[\ding{217}]
			\item reduce the errors created by the "fictitious" boundary 
			\item ensure the correct condition number of the finite element matrix
			\item permit to restore the coercivity of the bilinear scheme
		\end{enumerate}
	\end{frame}

	\begin{frame}{Non-homogeneous case}
		TO DO !
	\end{frame}

	\subsection{Fourier Neural Operator (FNO)}
	
	\begin{frame}{Presentation}
		\begin{enumerate}[\ding{217}]
			\item widely used in PDE solving and constitute an active field of research
			\item FNO are Neural Operator networks : Unlike standard neural networks, which learn using inputs and outputs of fixed dimensions, neural operators \textbf{learn operators, which are mappings between spaces of functions}. \item can be evaluated at any data resolution without the need for retraining
		\end{enumerate}
	\end{frame}

	\begin{frame}{Architecture of the FNO}
		\begin{center}
			\centering
			\pgfimage[width=\linewidth]{images/methods_tools/FNO_schema.png}
		\end{center}
		\textbf{Input $X$} of shape (bs,ni,nj,nk) \qquad \qquad \textbf{Output $Y$} of shape (bs,ni,nj,1) \\
		with bs the batch size, ni and nj the grid resolution and nk the number of channels.
	\end{frame}

	\begin{frame}{Description of the FNO architecture}
		\begin{center}
			\centering
			\pgfimage[width=\linewidth]{images/methods_tools/FNO_schema_moitie1.png}
		\end{center}
		\begin{enumerate}[\ding{217}]
			\item perform a $P$ transformation, to move to a space with more channels (to build a sufficiently rich representation of the data)
			\item apply $L$ Fourier layers defined by
			$$\mathcal{H}_\theta^l(\tilde{X})=\sigma\left(\mathcal{C}_\theta^l(\tilde{X})+\mathcal{B}_\theta^l(\tilde{X})\right),\; l=1,\dots,L$$
			with $\tilde{X}$ the input of the current layer and
			\begin{itemize}
				\item $\sigma$ an activation function (ReLU or GELU)
				\item $\mathcal{C}_\theta^l$ : convolution sublayer (convolution performed by Fast Fourier Transform)
				\item $\mathcal{B}_\theta^l$ : "bias-sublayer"
			\end{itemize}
			\item return to the target dimension by performing a $Q$ transformation (in our case, the number of output channels is 1)
		\end{enumerate}
	\end{frame}

	\begin{frame}{Fourier Layer Structure}
		\setstretch{0.5}
		\textbf{Convolution sublayer : } \quad $\mathcal{C}_\theta^l(X)=\mathcal{F}^{-1}(\mathcal{F}(X)\cdot\hat{W})$ \quad
		\begin{minipage}{0.3\linewidth}
			\vspace{-15pt}
			\centering
			\pgfimage[width=\linewidth]{images/methods_tools/FNO_schema_moitie2.png}
		\end{minipage}
		\begin{enumerate}[\ding{217}]
			\item $\hat{W}$ : a trainable kernel
			\item $\mathcal{F}$ : 2D Discrete Fourier Transform (DFT) defined by
			\begin{equation*}
				\mathcal{F}(X)_{ijk}=\frac{1}{ni}\frac{1}{nj}\sum_{i'=0}^{ni-1}\sum_{j'=0}^{nj-1}X_{i'j'k}e^{-2\sqrt{-1}\pi\left(\frac{ii'}{ni}+\frac{jj'}{nj}\right)}
			\end{equation*}
			$\mathcal{F}^{-1}$ : its inverse.
			\item $(Y\cdot\hat{W})_{ijk}=\sum_{k'}Y_{ijk'}\hat{W}_{ijk'} \quad \Rightarrow \quad$ applied channel by channel
		\end{enumerate} \; \\
		\textbf{Bias-sublayer :} \quad  $\mathcal{B}_\theta^l(X)_{ijk}=\sum_{k'}X_{ijk}W_{k'k}+B_k$ \quad
		\begin{minipage}{0.3\linewidth}
			\vspace{-10pt}
			\pgfimage[width=0.3\linewidth]{images/methods_tools/FNO_schema_moitie2_bis.png}
		\end{minipage}
		\begin{enumerate}[\ding{217}]
			\item 2D convolution with a kernel of size 1
			\item allowing channels to be mixed via a kernel without allowing interaction between pixels.
		\end{enumerate}
	\end{frame}

	\begin{frame}{Some details on the FNO}
		\begin{enumerate}[\ding{217}]
			\item \textbf{Mesh resolution independent : }
			\begin{itemize}
				\item $P$ and $Q$ = fully-connected multi-layer perceptron $\Rightarrow$ perform local transformations at each point
				\item Fourier layers also independent of mesh resolution : learn in Fourier space so the value of the Fourier modes does not depend on the mesh resolution
			\end{itemize}
			\item \textbf{Low pass filter :} truncate high Fourier modes to ignore high frequencies $\Rightarrow$ enable a kind of regularization that helps the generalization
			\begin{center}
				\pgfimage[width=0.6\linewidth]{images/methods_tools/FNO_low_pass_filter.png}
			\end{center}
		\end{enumerate}		
	\end{frame}

	\begin{frame}{Application}
		TO DO !
	\end{frame}
	
	\section{Correction}
	
	\begin{frame}{Problems considered}
		\setstretch{0.5}
		\small 
		\textbf{1st problem considered :} Trigonometric solution on a Square. \\
		\begin{minipage}{0.2\linewidth}
			\pgfimage[width=\linewidth]{images/corr/geom_square.png}
		\end{minipage} \;
		\begin{minipage}{0.78\linewidth}
			\begin{enumerate}[\ding{217}]
				\item Level-set function (for formulation) : $\phi(x,y)=x(1-x)y(1-y)$
				\item Level-set function (for construction) : $\phi_C(X)=||X-0.5||_\infty-0.5$
				\item Analytical solution : $u_{ex}(x,y)=S\times sin\left(2\pi fx+\varphi\right)\times sin\left(2\pi fy+\varphi\right)$
				\begin{itemize}
					\item $S\in[0,1]$ : amplitude of the signal 
					\item $f\in\mathbb{N}$ : "frequency" of the signal
					\item $\varphi\in[0,1]$ : phase at the origin
				\end{itemize}
				\item Source term : $f(x,y)=8\pi^2 Sf^2\sin\left(2\pi fx + \varphi\right)\sin\left(2\pi fy + \varphi\right)$
			\end{enumerate}
		\end{minipage} \; \\ \; \\
		
		\textbf{2nd problem considered :} Unknown solution on a Circle. \\
		\begin{minipage}{0.2\linewidth}
			\pgfimage[width=\linewidth]{images/corr/geom_circle.png}
		\end{minipage} \;
		\begin{minipage}{0.78\linewidth}
			\begin{enumerate}[\ding{217}]
				\item Level-set function : $\phi(x,y)=-1/8+(x-1/2)^2+(y-1/2)^2$
				\item Source term : $f(x,y) = \exp\left(-\frac{(x-\mu_0)^2 + (y-\mu_1)^2}{2\sigma^2}\right)$
				\begin{itemize}
					\item $\sigma \sim \mathcal{U}([0.1,0.6])$
					\item $\mu_0, \mu_1 \sim \mathcal{U}([0.5-\sqrt{2}/4, 0.5+\sqrt{2}/4])$ 
				\end{itemize}
				\item a reference solution $u_{ref}$ : over-refined $\mathbb{P}^1$ solution (with FEM)
			\end{enumerate}
		\end{minipage}
	\end{frame}
	
	\subsection{Methods considered}
	
	\begin{frame}{Correction by adding}
		We are given $\tilde{\phi}$ an "initial" solution to the problem under consideration.
		
		We will consider
		\begin{equation*}
			\tilde{u}=\tilde{\phi}+\tilde{C}
		\end{equation*}
	
		We want to find $\tilde{C}: \Omega \rightarrow \mathbb{R}^d$ solution to the problem
		\begin{equation*}
			\left\{\begin{aligned}
				-\Delta \tilde{u}&=f, \; &&\text{on } \Omega, \\
				\tilde{u}&=g, \; &&\text{in } \Gamma.
			\end{aligned}\right.
		\end{equation*}
	
		Rewriting the problem, we seek to find $\tilde{C}: \Omega \rightarrow \mathbb{R}^d$ solution to the problem
		\begin{equation*}
			\left\{\begin{aligned}
				-\Delta \tilde{C}&=\tilde{f}, \; &&\text{on } \Omega, \\
				\tilde{C}&=0, \; &&\text{in } \Gamma.
			\end{aligned}\right. %\tag{$\mathcal{C}_{+}$}
		\end{equation*}
		with $\tilde{f}=f+\Delta\tilde{\phi}$.

		In practice, it may be useful to integrate by parts the term containing $\Delta \tilde{\phi}$.
	\end{frame}

	\begin{frame}{Correction by multiplying}
		We will considering 
		\begin{equation*}
			\tilde{u}=\tilde{\phi}C
		\end{equation*}
		
		We want to find $C: \Omega \rightarrow \mathbb{R}^d$ solution to the problem
		\begin{equation*}
			\left\{\begin{aligned}
				&-\Delta (\tilde{\phi}C)=f, \; &&\text{on } \Omega, \\
				&C=1, \; &&\text{on } \; \Gamma.
			\end{aligned}\right.
		\end{equation*}
		
		In the non-homogeneous case, it is important to impose the boundary conditions either by the direct method or by the dual method.
	\end{frame}

	\begin{frame}{Correction by multiplying (elevated problem)}
		We introduced an initial modified problem : Find $\hat{u} : \Omega \rightarrow \mathbb{R}^d$ such that
		\begin{equation*}
			\left\{
			\begin{aligned}
				-\Delta \hat{u} = f, \; &&\text{in } \; \Omega, \\
				\hat{u}=g+m, \; &&\text{on } \; \Gamma,
			\end{aligned}
			\right.
		\end{equation*}
		with $\hat{u}=u+m$ and $m$ a constant.
		
		We then apply the multiplication correction on the elevated problem by considering
		\begin{equation*}
			\hat{\phi}=\tilde{\phi}+m
		\end{equation*}
		and so we look for $C: \Omega \rightarrow \mathbb{R}^d$ solution to the problem
		\begin{equation*}
			\label{eq.corr.pbc_mult_reh}
			\left\{\begin{aligned}
				&-\Delta (\hat{\phi}C)=f, \; &&\text{in } \; \Omega, \\
				&C=1, \; &&\text{on } \; \Gamma.
			\end{aligned}\right. %\tag{$\mathcal{C}_\times^\mathcal{M}$}
		\end{equation*}
		
		In the case of this correction, it is important to impose the boundary conditions either by the direct method, or by the dual method.
	\end{frame}
%	
	\subsection{Theoretical results}
	
	\subsection{Numerical results}
	
	\section{Conclusion} %perspectives

	\section{Bibliography}
	
	\begin{frame}[allowframebreaks]{}
		\printbibliography[heading=none]
	\end{frame}
	
\end{document}
