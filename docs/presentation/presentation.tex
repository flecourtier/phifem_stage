%% Requires compilation with XeLaTeX or LuaLaTeX
\documentclass[compress,10pt,xcolor={table,dvipsnames},t]{beamer}
%\documentclass[compress]{beamer}
\usetheme{diapo}
\usepackage{amsmath}
\usepackage[bottom]{footmisc}
\usepackage{multirow}
\usepackage{setspace}
\usepackage{caption}
\usepackage{array,multirow,makecell}
\usepackage[table]{xcolor}
\usepackage{pifont}
\usepackage{hyperref}
\usepackage[utf8]{inputenc}
\setcellgapes{1pt}
\setlength{\parindent}{0pt}
\makegapedcells
\newcolumntype{R}[1]{>{\raggedleft\arraybackslash }b{#1}}
\newcolumntype{L}[1]{>{\raggedright\arraybackslash }b{#1}}
\newcolumntype{C}[1]{>{\centering\arraybackslash }b{#1}}
\usepackage{paralist}

\usepackage[backend=biber,style=numeric,sorting=nyt]{biblatex}
\renewcommand*{\bibfont}{\scriptsize}
\addbibresource{biblio.bib}

\hypersetup{
	colorlinks=true,
	urlcolor=blue,
	citecolor=blue,
	linkcolor=titre,
}

\title[PhiFEM]{Innovative non-conformal finite element methods for augmented surgery}
\subtitle{Internship presentation}
\author[name]{LECOURTIER Frédérique, DUPREZ Michel, FRANCK Emmanuel, LLERAS Vanessa}
\institute{\large Strasbourg University}
\date{\today}

\useoutertheme[subsection=false]{miniframes}
\usepackage{etoolbox}
\makeatletter
\patchcmd{\slideentry}{\advance\beamer@xpos by1\relax}{}{}{}
\def\beamer@subsectionentry#1#2#3#4#5{\advance\beamer@xpos by1\relax}%
\makeatother

\allowbreak

\begin{document}
	\nocite{*}
	
	\begin{frame}
		\vspace{-20pt}
		\titlepage
	\end{frame}
	
	\AtBeginSection[]{
		\begin{frame}
			\vfill
			\centering
			\begin{beamercolorbox}[sep=5pt,shadow=true,rounded=true]{subtitle}
				\usebeamerfont{title}\insertsectionhead\par%
			\end{beamercolorbox}
			%\tableofcontents[sectionstyle=hide,subsectionstyle=show]
			\tableofcontents[sectionstyle=hide,subsectionstyle=show/shaded/hide]
			\vfill
		\end{frame}
	}

	\AtBeginSubsection[]{
		\begin{frame}
			\vfill
			\centering
			\begin{beamercolorbox}[sep=5pt,shadow=true,rounded=true]{subtitle}
				\usebeamerfont{title}\insertsectionhead\par%
			\end{beamercolorbox}
			\tableofcontents[sectionstyle=hide,subsectionstyle=show/shaded/hide]
			\vfill
		\end{frame}
	}

	\section{Introduction}

	\begin{frame}{Presentation of the teams}
		\begin{center}
			\pgfimage[width=0.3\linewidth]{images/intro/logo-mimesis.png}
		\end{center}
		\begin{enumerate}[\ding{217}]
			\item project-team as sub-team of MLMS ("Machine Leraning, Modélisation et Simulation") of Inria
			\item \textbf{Aim :} to create real-time digital twins of an organ
			\item \textbf{Scientific challenges :}
			\begin{itemize}
				\item scientific computing
				\item data assimilation
				\item machine learning
				\item control
			\end{itemize}
			\item \textbf{Main application domains :}
			\begin{itemize}
				\item surgical training
				\item surgical guidance during complex interventions
			\end{itemize}
		\end{enumerate}
	\end{frame}

	\begin{frame}{Scientific context}
		\color{red}A MODIFIER !!\color{black} \\
		Because of the geometry of organs, mimesis has developed a new method: the $\phi$-FEM method. \\
		\textbf{Abstract :} Mimesis propose a new fictitious domain finite element method, well suited for elliptic problems posed
		in a domain given by a level-set function without requiring a mesh fitting the boundary. 
		\begin{center}
			\pgfimage[width=0.8\linewidth]{images/intro/context_geometry.jpg}
		\end{center}
	\end{frame}

	\begin{frame}{Objectives - Deliverables}
		\underline{\textbf{Objectives :}} \\
		\begin{enumerate}[\ding{217}]
			\item Train a Fourier Neural Operator (FNO), with $\phi$-FEM solutions, to predict the solutions of a given PDE.
			\item Apply a correction on the FNO predictions.
		\end{enumerate}
		\textbf{Aim :} Neural networks are very fast, but not very accurate \\
		=> finite element methods are used to improve prediction accuracy.
		\textbf{Remark :} Implementation in Python with FEniCS, Pytorch and Tensorflow.
		\underline{\textbf{Deliverables :}}

		\begin{enumerate}[\ding{217}]
			\item a \href{https://github.com/flecourtier/phifem_stage/blob/main/docs/suivi/suivi.pdf}{weekly tracking report} ( in French)
			\item a \href{https://github.com/flecourtier/phifem_stage}{github repository} containing all the code allowing to reproduce the results presented in this report
			\item a \href{https://csmi.cemosis.fr/csmi-stages-2023/m2/_attachments/Lecourtier-Fr\%C3\%A9d\%C3\%A9rique.pdf}{report} of the internship
			\item an \href{https://flecourtier.github.io/phifem_stage/phifem_project/1.0.3/main_page.html}{online report} generated with a tool called antora (made by a github CI)
		\end{enumerate}
	\end{frame}

	\begin{frame}{Problem considered}
		\textbf{Poisson problem with Dirichlet conditions :} \\
		Find $u : \Omega \rightarrow \mathbb{R}^d (d=1,2,3)$ such that
		\begin{equation}
			\left\{
			\begin{aligned}
				-\Delta u &= f, \; &&\text{in } \; \Omega, \\
				u&=g, \; &&\text{on } \; \partial\Omega,
			\end{aligned}
			\right. \tag{$\mathcal{P}$} \label{pb_initial}
		\end{equation}
		with $\Delta$ the Laplace operator, $\Omega$ a smooth bounded open set and $\partial\Omega$ its boundary. \\
		In this section, we defined by
		\begin{equation*}
			||u_{ex}-u_{method}||_{0,\Omega_h}^{(rel)}=\frac{\int_{\Omega_h} (u_{ex}-u_{method})^2}{\int_{\Omega_h} u_{ex}^2}
		\end{equation*}
		the relative error between the exact solution $u_{ex}$ and $u_{method}$ a solution obtained by FEM or $\phi$-FEM, a correction solver or the prediction of an neural network.
		
%		We can defined too
%		\begin{equation*}
%			||u_{ex}-u_{method}||_{0,\Omega_h}^{(abs)}=\int_{\Omega_h} (u_{ex}-u_{method})^2
%		\end{equation*}
%		the absolute error.
	\end{frame}

	\section{General methods and tools}
	
	\subsection{Standard FEM method}
	
	\begin{frame}{Presentation of standard FEM method}	
		\textbf{Variational Problem :} 
		\begin{equation*}
			\text{Find } u\in V \text{ such that } a(u,v)=l(v), \;\forall v\in V
		\end{equation*}
		where $V$ is a Hilbert space, $a$ is a bilinear form and $l$ is a linear form.
		
		\textbf{Approach Problem :} 
		\begin{equation*}
			\text{Find } u_h\in V_h \text{ such that } a(u_h,v_h)=l(v_h), \;\forall v_h\in V
		\end{equation*}
		with $u_h$ an approximate solution in $V_h$, a finite-dimensional space dependent on $h$ such that $\quad V_h\subset V, \; dimV_h = N_h<\infty, \; \forall h>0$ 
		
		As $u_h=\sum_{i=1}^{N_h}u_i\varphi_i$ with $(\varphi_1,\dots,\varphi_{N_h})$ a basis of $V_h$, finding an approximation of the PDE solution implies solving the following linear system:
		\begin{equation*}
			AU=b
		\end{equation*}
		with
		\begin{equation*}
			A=(a(\varphi_i,\varphi_j))_{1\le i,j\le N_h}, \quad U=(u_i)_{1\le i\le N_h} \quad \text{and} \quad b=(l(\varphi_j))_{1\le j\le N_h}
		\end{equation*}
	\end{frame}

	\begin{frame}{In practice}
		\begin{enumerate}[\ding{217}]
			\item \begin{minipage}[t]{0.68\linewidth}
				Construct a mesh of our $\Omega$ geometry with a family of elements (in 2D: triangle, rectangle; in 3D: tetrahedron, parallelepiped, prism) defined by
				$$\mathcal{T}_h = \left\{K_1,\dots,K_{N_e}\right\}$$
				where $N_e$ is the number of elements. \\
			\end{minipage} \begin{minipage}[t][][b]{0.28\linewidth}
				\centering
				\qquad \pgfimage[width=0.8\linewidth]{images/methods_tools/FEM_triangle_mesh.png}
			\end{minipage}
			\item Construct a space of piece-wise affine continuous functions, defined by
			\begin{equation*}
				V_h:=P_{C,h}^k=\{v_h\in C^0(\bar{\Omega}), \forall K\in\mathcal{T}_h, {v_h}_{|K}\in\mathbb{P}_k\}
			\end{equation*}
			where $\mathbb{P}_k$ is a vector space of polynomials of total degree less than or equal to $k$.
			\item In the case of non-homogeneous condition : use of penalization or elimination methods.
		\end{enumerate}
		
	\end{frame}

	\subsection{$\phi$-FEM method}
	
	\begin{frame}{Context}
		\textbf{Idea :} $\phi$-FEM method = new fictitious domain finite element method that does not require a mesh conforming to the real boundary.
		\begin{center}
			\pgfimage[width=0.7\linewidth]{images/methods_tools/PhiFEM_context_mesh.png}
		\end{center}
		\textbf{Advantage :} boundary represented by a level-set function $\Rightarrow$ only this function will change over time during real-time simulation
	\end{frame}
	
	\begin{frame}{Problem}
		We pose $u=\phi w$ such that
		$$\left\{\begin{aligned}
			-\Delta (\phi w) &= f, \; \text{on } \Omega, \\
			u&=g, \; \text{in } \Gamma, \\
		\end{aligned}\right.$$
		where $\phi$ is the level-set function and $\Omega$ and $\Gamma$ are given by :
		$$
		\Omega=\{\phi < 0\} \quad \text{and} \quad \Gamma=\{\phi = 0\}.$$
		\begin{center}
			\pgfimage[width=0.3\linewidth]{images/methods_tools/PhiFEM_level_set.png}
		\end{center}
		The level-set function $\phi$ is supposed to be known on $\mathbb{R}^d$ and sufficiently smooth. \\
		For instance, the signed distance to $\Gamma$ is a good candidate
	\end{frame}
	
	\begin{frame}{Fictitious domain}
		\setstretch{0.5}
		\begin{minipage}{0.39\linewidth}
			\centering
			\pgfimage[width=\linewidth]{images/methods_tools/PhiFEM_domain.png} \\ \tiny \; \\
			\pgfimage[width=0.15\linewidth]{images/methods_tools/PhiFEM_fleche.png} \qquad \qquad \qquad \qquad  \\
			\tiny \pgfimage[width=\linewidth]{images/methods_tools/PhiFEM_domain_considered.png}
		\end{minipage}
		\begin{minipage}{0.6\linewidth}
			\begin{enumerate}[\ding{217}]
				\item $\mathcal{O}$ : fictitious domain such that $\Omega\subset\mathcal{O}$
				\item $\mathcal{T}_h^\mathcal{O}$ : simple quasi-uniform mesh on $\mathcal{O}$
				\item $\phi_h=I_{h,\mathcal{O}}^{(l)}(\phi)\in V_{h,\mathcal{O}}^{(l)}$ : approximation of $\phi$ \\ 
				with $I_{h,\mathcal{O}}^{(l)}$ the standard Lagrange interpolation operator on
				$$V_{h,\mathcal{O}}^{(l)}=\left\{v_h\in H^1(\mathcal{O}):v_{h|_T}\in\mathbb{P}_l(T) \;  \forall T\in\mathcal{T}_h^\mathcal{O}\right\}$$
				\item $\Gamma_h=\{\phi_h=0\}$ : approximate boundary of $\Gamma$
				\item $\mathcal{T}_h$ : sub-mesh of $\mathcal{T}_h^\mathcal{O}$ defined by
				$$\mathcal{T}_h=\left\{T\in \mathcal{T}_h^\mathcal{O}:T\cap\{\phi_h<0\}\ne\emptyset\right\}$$
				\item $\Omega_h$ : domain covered by the $\mathcal{T}_h$ mesh defined by
				$$\Omega_h=\left(\cup_{T\in\mathcal{T}_h}T\right)^O$$
				($\partial\Omega_h$ its boundary)
			\end{enumerate}			
		\end{minipage}
	\end{frame}
	
	\begin{frame}{Facets and Cells sets}
		\begin{minipage}{0.38\linewidth}
			\centering
			\pgfimage[width=\linewidth]{images/methods_tools/PhiFEM_boundary_cells.png} \\ \; \\ \; \\
			\pgfimage[width=\linewidth]{images/methods_tools/PhiFEM_boundary_edges.png}
		\end{minipage} \;
		\begin{minipage}{0.6\linewidth}
			\begin{enumerate}[\ding{217}]
				\item $\mathcal{T}_h^\Gamma\subset \mathcal{T}_h$ : contains the mesh elements cut by $\Gamma_h$, i.e. 
				\begin{equation*}
					\mathcal{T}_h^\Gamma=\left\{T\in\mathcal{T}_h:T\cap\Gamma_h\ne\emptyset\right\},
				\end{equation*}
				\item $\Omega_h^\Gamma$ : domain covered by the $\mathcal{T}_h^\Gamma$ mesh, i.e.
				\begin{equation*}
					\Omega_h^\Gamma=\left(\cup_{T\in\mathcal{T}_h^\Gamma}T\right)^O
				\end{equation*}
				\item $\mathcal{F}_h^\Gamma$ : collects the interior facets of $\mathcal{T}_h$ either cut by $\Gamma_h$ or belonging to a cut mesh element, i.e.
				\begin{align*}
					\mathcal{F}_h^\Gamma=\left\{E\;(\text{an internal facet of } \mathcal{T}_h) \text{ such that }\right. \\
					\left. \exists T\in \mathcal{T}_h:T\cap\Gamma_h\ne\emptyset \text{ and } E\in\partial T\right\}
				\end{align*}
			\end{enumerate}
		\end{minipage}
	\end{frame}

	\begin{frame}{Application to the Poisson problem}
		We start by consider the \textbf{homogeneous case} ($g=0$ on $\Gamma$). \\
		\textbf{Approach Problem :} Find $w_h\in V_h^{(k)}$ such that 
		$$a_h(w_h,v_h) = l_h(v_h) \quad \forall v_h \in V_h^{(k)}$$
		where
		$$a_h(w,v)=\int_{\Omega_h} \nabla (\phi_h w) \cdot \nabla (\phi_h v) - \int_{\partial\Omega_h} \frac{\partial}{\partial n}(\phi_h w)\phi_h v+G_h(w,v),$$
		$$l_h(v)=\int_{\Omega_h} f \phi_h v + G_h^{rhs}(v)$$
		and 
		$$V_h^{(k)}=\left\{v_h\in H^1(\Omega_h):v_{h|_T}\in\mathbb{P}_k(T), \; \forall T\in\mathcal{T}_h\right\}.$$
	\end{frame}

	\begin{frame}{Stabilization terms}
		\begin{center}
			\centering
			\pgfimage[width=\linewidth]{images/methods_tools/PhiFEM_stab_terms.png}
		\end{center}
		\small
		\underline{1st term :} Ghost penality \cite{burman_ghost_2010}, ensure continuity of the solution by penalizing gradient jumps. \\
		\underline{2nd term :} require the solution to verify the strong form on $\Omega_h^\Gamma$. \\
		\normalsize
		\textbf{Purpose :} 
		\begin{enumerate}[\ding{217}]
			\item reduce the errors created by the "fictitious" boundary 
			\item ensure the correct condition number of the finite element matrix
			\item permit to restore the coercivity of the bilinear scheme
		\end{enumerate}
	\end{frame}

	\begin{frame}{Non-homogeneous case - Direct method}
		\setstretch{0.5}
		\small
		Suppose that $g$ is currently given over the entire $\Omega_h$, we have
		\begin{equation*}
			u=\phi w +g, \; \text{on } \Omega_h.
		\end{equation*}
		\textit{Problem :} Find $w_h$ on $\Omega_h$ such that
		\begin{align*}
			\int_{\Omega_h}\nabla(\phi_h w_h)&\nabla(\phi_h v_h)-\int_{\partial\Omega_h}\frac{\partial}{\partial n}(\phi_h w_h)\phi_h v_h+G_h(w_h,v_h)=\int_{\Omega_h}f\phi_h v_h \\
			&-\int_{\Omega_h}\nabla g\nabla(\phi_h v_h)+\int_{\partial\Omega_h}\frac{\partial g}{\partial n}\phi_h v_h+G_h^{rhs}(v_h), \; \forall v_h\in \Omega_h
		\end{align*}
		with
		\begin{equation*}
			G_h(w,v)=\sigma h\sum_{E\in\mathcal{F}_h^\Gamma}\int_E\left[\frac{\partial}{\partial n}(\phi_h w)\right]\left[\frac{\partial}{\partial n}(\phi_h v)\right]+\sigma h^2\sum_{T\in\mathcal{T}_h^\Gamma}\int_T \Delta(\phi_h w)\Delta(\phi_h v)
		\end{equation*}
		and
		\begin{equation*}
			G_h^{rhs}(v)=-\sigma h^2\sum_{T\in\mathcal{T}_h^\Gamma}\int_T f\Delta(\phi_h v)-\sigma h\sum_{E\in\mathcal{F}_h^\Gamma}\int_E\left[\frac{\partial g}{\partial n}\right]\left[\frac{\partial}{\partial n}(\phi_h v)\right]-\sigma h^2\sum_{T\in\mathcal{T}_h^\Gamma}\int_T \Delta g\Delta(\phi_h v)
		\end{equation*}
	\end{frame}
			
	\begin{frame}{Non-homogeneous case - Dual method}
		\setstretch{0.5}
		\small
		Assuming $g$ defined on $\Omega_h^\Gamma$, introduce $p$ on $\Omega_h^\Gamma$ in addition to $u$ on $\Omega_h$ with
		\begin{equation*}
			u=\phi p+g, \; \text{on } \Omega_h^\Gamma.
		\end{equation*}
		\textit{Problem :} Find $u$ on $\Omega_h$ and $p$ on $\Omega_h^\Gamma$ such that
		\begin{align*}
			\int_{\Omega_h}\nabla u\nabla v&-\int_{\partial\Omega_h}\frac{\partial u}{\partial n} v + \frac{\gamma}{h^2} \sum_{T\in\mathcal{T}_h^\Gamma}\int_T \left(u-\frac{1}{h}\phi p\right)\left(v-\frac{1}{h}\phi q\right) + G_h(u,v) = \int_{\Omega_h}fv \\
			&+ \frac{\gamma}{h^2} \sum_{T\in\mathcal{T}_h^\Gamma}\int_T g\left(v-\frac{1}{h}\phi q\right) + G_h^{rhs}(v), \; \forall v \; \text{on } \Omega_h, \; q \; \text{on } \Omega_h^\Gamma.
		\end{align*}
		with $\gamma$ an other positive stabilization parameter,
		\begin{equation*}
			G_h(u,v)=\sigma h\sum_{E\in\mathcal{F}_h^\Gamma}\int_E\left[\frac{\partial u}{\partial n}\right]\left[\frac{\partial v}{\partial n}\right]+\sigma h^2\sum_{T\in\mathcal{T}_h^\Gamma}\int_T \Delta u\Delta v
		\end{equation*}
		\begin{equation*}
			\text{and } \qquad G_h^{rhs}(v)=-\sigma h^2\sum_{T\in\mathcal{T}_h^\Gamma}\int_T f\Delta v.
		\end{equation*}
	\end{frame}

	\subsection{Fourier Neural Operator (FNO)}
	
	\begin{frame}{Presentation}
		\begin{enumerate}[\ding{217}]
			\item widely used in PDE solving and constitute an active field of research
			\item FNO are Neural Operator networks : Unlike standard neural networks, which learn using inputs and outputs of fixed dimensions, neural operators \textbf{learn operators, which are mappings between spaces of functions}. \item can be evaluated at any data resolution without the need for retraining
		\end{enumerate}
	\end{frame}

	\begin{frame}{Architecture of the FNO}
		\begin{center}
			\centering
			\pgfimage[width=\linewidth]{images/methods_tools/FNO_schema.png}
		\end{center}
		\textbf{Input $X$} of shape (bs,ni,nj,nk) \qquad \qquad \textbf{Output $Y$} of shape (bs,ni,nj,1) \\
		with bs the batch size, ni and nj the grid resolution and nk the number of channels.
	\end{frame}

	\begin{frame}{Description of the FNO architecture}
		\begin{center}
			\centering
			\pgfimage[width=\linewidth]{images/methods_tools/FNO_schema_moitie1.png}
		\end{center}
		\begin{enumerate}[\ding{217}]
			\item perform a $P$ transformation, to move to a space with more channels (to build a sufficiently rich representation of the data)
			\item apply $L$ Fourier layers defined by
			$$\mathcal{H}_\theta^l(\tilde{X})=\sigma\left(\mathcal{C}_\theta^l(\tilde{X})+\mathcal{B}_\theta^l(\tilde{X})\right),\; l=1,\dots,L$$
			with $\tilde{X}$ the input of the current layer and
			\begin{itemize}
				\item $\sigma$ an activation function (ReLU or GELU)
				\item $\mathcal{C}_\theta^l$ : convolution sublayer (convolution performed by Fast Fourier Transform)
				\item $\mathcal{B}_\theta^l$ : "bias-sublayer"
			\end{itemize}
			\item return to the target dimension by performing a $Q$ transformation (in our case, the number of output channels is 1)
		\end{enumerate}
	\end{frame}

	\begin{frame}{Fourier Layer Structure}
		\setstretch{0.5}
		\textbf{Convolution sublayer : } \quad $\mathcal{C}_\theta^l(X)=\mathcal{F}^{-1}(\mathcal{F}(X)\cdot\hat{W})$ \quad
		\begin{minipage}{0.3\linewidth}
			\vspace{-15pt}
			\centering
			\pgfimage[width=\linewidth]{images/methods_tools/FNO_schema_moitie2.png}
		\end{minipage}
		\begin{enumerate}[\ding{217}]
			\item $\hat{W}$ : a trainable kernel
			\item $\mathcal{F}$ : 2D Discrete Fourier Transform (DFT) defined by
			\begin{equation*}
				\mathcal{F}(X)_{ijk}=\frac{1}{ni}\frac{1}{nj}\sum_{i'=0}^{ni-1}\sum_{j'=0}^{nj-1}X_{i'j'k}e^{-2\sqrt{-1}\pi\left(\frac{ii'}{ni}+\frac{jj'}{nj}\right)}
			\end{equation*}
			$\mathcal{F}^{-1}$ : its inverse.
			\item $(Y\cdot\hat{W})_{ijk}=\sum_{k'}Y_{ijk'}\hat{W}_{ijk'} \quad \Rightarrow \quad$ applied channel by channel
		\end{enumerate} \; \\
		\textbf{Bias-sublayer :} \quad  $\mathcal{B}_\theta^l(X)_{ijk}=\sum_{k'}X_{ijk}W_{k'k}+B_k$ \quad
		\begin{minipage}{0.3\linewidth}
			\vspace{-10pt}
			\pgfimage[width=0.3\linewidth]{images/methods_tools/FNO_schema_moitie2_bis.png}
		\end{minipage}
		\begin{enumerate}[\ding{217}]
			\item 2D convolution with a kernel of size 1
			\item allowing channels to be mixed via a kernel without allowing interaction between pixels.
		\end{enumerate}
	\end{frame}

	\begin{frame}{Some details on the FNO}
		\begin{enumerate}[\ding{217}]
			\item \textbf{Mesh resolution independent : }
			\begin{itemize}
				\item $P$ and $Q$ = fully-connected multi-layer perceptron $\Rightarrow$ perform local transformations at each point
				\item Fourier layers also independent of mesh resolution : learn in Fourier space so the value of the Fourier modes does not depend on the mesh resolution
			\end{itemize}
			\item \textbf{Low pass filter :} truncate high Fourier modes to ignore high frequencies $\Rightarrow$ enable a kind of regularization that helps the generalization
			\begin{center}
				\pgfimage[width=0.5\linewidth]{images/methods_tools/FNO_low_pass_filter.png}
			\end{center}
		\end{enumerate}		
		\textbf{Limitation :} Use of regular grid $\Rightarrow$ only $\mathbb{P}_1$ or $\mathbb{P}_2$ solution
	\end{frame}

	\begin{frame}{Application}
		FNO Training :
		\begin{center}
			\pgfimage[width=\linewidth]{images/methods_tools/FNO_train_schema.png}
		\end{center}
		Correction on the FNO predictions :
		\begin{center}
			\pgfimage[width=\linewidth]{images/methods_tools/FNO_test_schema.png}
		\end{center}
	\end{frame}
	
	\section{Correction}
	
	\subsection{Problems considered}
	
	\begin{frame}{Trigonometric solution on a Square}
		\setstretch{0.7}
		\begin{minipage}{0.3\linewidth}
			\centering
			\pgfimage[width=0.8\linewidth]{images/corr/geom_square.png}
		\end{minipage} \;
		\begin{minipage}{0.68\linewidth}
			\begin{enumerate}[\ding{217}]
				\item Level-set function (for formulation) : $$\phi(x,y)=x(1-x)y(1-y)$$
				\item Level-set function (for construction) : $$\phi_C(X)=||X-0.5||_\infty-0.5$$
				
			\end{enumerate}
		\end{minipage}

		
			\begin{enumerate}[\ding{217}]
				\item Analytical solution : (Homogeneous if $\varphi=0$)
				$$u_{ex}(x,y)=S\times sin\left(2\pi fx+\varphi\right)\times sin\left(2\pi fy+\varphi\right)$$
				\item Source term : $$f(x,y)=8\pi^2 Sf^2sin\left(2\pi fx + \varphi\right)sin\left(2\pi fy + \varphi\right)$$
				\begin{minipage}{0.48\linewidth}
					\begin{itemize}
						\item $S\in[0,1]$ : amplitude of the signal 
						\item $f\in\mathbb{N}$ : "frequency" of the signal
						\item $\varphi\in[0,1]$ : phase at the origin
					\end{itemize}
				\end{minipage} \;
				\begin{minipage}{0.48\linewidth}
					\centering
					\pgfimage[width=\linewidth]{images/corr/norms.png}
				\end{minipage}
			\end{enumerate}
	\end{frame}

	\begin{frame}{Unknown solution on a Circle}
		\setstretch{0.7}
		\begin{minipage}{0.3\linewidth}
			\centering
			\pgfimage[width=0.8\linewidth]{images/corr/geom_circle.png}
		\end{minipage} \;
		\begin{minipage}{0.68\linewidth}
			\begin{enumerate}[\ding{217}]
				\item Level-set function : 
				$$\phi(x,y)=-1/8+(x-1/2)^2+(y-1/2)^2$$
			\end{enumerate}
		\end{minipage}
		\begin{enumerate}[\ding{217}]
			\item Source term : 
			$$f(x,y) = exp\left(-\frac{(x-\mu_0)^2 + (y-\mu_1)^2}{2\sigma^2}\right)$$
			\begin{itemize}
				\item $\sigma \sim \mathcal{U}([0.1,0.6])$
				\item $\mu_0, \mu_1 \sim \mathcal{U}([0.5-\sqrt{2}/4, 0.5+\sqrt{2}/4])$ 
			\end{itemize}
			with the condition $\phi(\mu_0, \mu_1) < -0.05$
			\item a reference solution $u_{ref}$ : over-refined $\mathbb{P}^1$ solution (with FEM)
		\end{enumerate}
	\end{frame}
	
	\subsection{Methods considered}
	
	\begin{frame}{Correction by adding}
		We are given $\tilde{\phi}$ an "initial" solution to the problem under consideration.
		
		We will consider
		\begin{equation*}
			\tilde{u}=\tilde{\phi}+\tilde{C}
		\end{equation*}
	
		We want to find $\tilde{C}: \Omega \rightarrow \mathbb{R}^d$ solution to the problem
		\begin{equation*}
			\left\{\begin{aligned}
				-\Delta \tilde{u}&=f, \; &&\text{on } \Omega, \\
				\tilde{u}&=g, \; &&\text{in } \Gamma.
			\end{aligned}\right.
		\end{equation*}
		with $\tilde{C}\phi C$ for the $\phi$-FEM method
		Rewriting the problem, we seek to find $\tilde{C}: \Omega \rightarrow \mathbb{R}^d$ solution to the problem
		\begin{equation}
			\left\{\begin{aligned}
				-\Delta \tilde{C}&=\tilde{f}, \; &&\text{on } \Omega, \\
				\tilde{C}&=0, \; &&\text{in } \Gamma.
			\end{aligned}\right. \tag{$\mathcal{C}_{+}$} \label{corr_add}
		\end{equation}
		with $\tilde{f}=f+\Delta\tilde{\phi}$.

		In practice, it may be useful to integrate by parts the term containing $\Delta \tilde{\phi}$.
	\end{frame}

	\begin{frame}{Correction by multiplying}
		We will considering 
		\begin{equation*}
			\tilde{u}=\tilde{\phi}C
		\end{equation*}
		
		We want to find $C: \Omega \rightarrow \mathbb{R}^d$ solution to the problem
		\begin{equation*}
			\left\{\begin{aligned}
				&-\Delta (\tilde{\phi}C)=f, \; &&\text{on } \Omega, \\
				&C=1, \; &&\text{on } \; \Gamma.
			\end{aligned}\right. \tag{$\mathcal{C}_\times$} \label{corr_mult}
		\end{equation*}
		
		In the non-homogeneous case, it is important to impose the boundary conditions either by the direct method or by the dual method.
	\end{frame}

	\begin{frame}{Correction by multiplying (elevated problem)}
		We introduced an initial modified problem : Find $\hat{u} : \Omega \rightarrow \mathbb{R}^d$ such that
		\begin{equation}
			\left\{
			\begin{aligned}
				-\Delta \hat{u} = f, \; &&\text{in } \; \Omega, \\
				\hat{u}=g+m, \; &&\text{on } \; \Gamma,
			\end{aligned}
			\right. \tag{$\mathcal{P}^\mathcal{M}$} \label{pb_reh}
		\end{equation}
		with $\hat{u}=u+m$ and $m$ a constant.
		
		We then apply the multiplication correction on the elevated problem by considering
		\begin{equation*}
			\hat{\phi}=\tilde{\phi}+m
		\end{equation*}
		and so we look for $C: \Omega \rightarrow \mathbb{R}^d$ solution to the problem
		\begin{equation*}
			\left\{\begin{aligned}
				&-\Delta (\hat{\phi}C)=f, \; &&\text{in } \; \Omega, \\
				&C=1, \; &&\text{on } \; \Gamma.
			\end{aligned}\right. \tag{$\mathcal{C}_\times^\mathcal{M}$} \label{corr_mult_reh}
		\end{equation*}
		
		In the case of this correction, it is important to impose the boundary conditions either by the direct method, or by the dual method.
	\end{frame}
	
	\begin{frame}{Theoretical results}
		Here, we're interested by (\ref{corr_mult_reh}) with standard FEM.
		\begin{enumerate}[\ding{217}]
			\item We can prove the following property:
			\begin{equation*}
				\left|\left|\hat{u_{ex}}-\hat{u_h}\right|\right|_0\le ch^{k+1}||\hat{\phi}||_\infty\left|C\right|_{k+1}
			\end{equation*}
			with $\hat{u_{ex}}=u_{ex}+m$ the exact solution of (\ref{pb_reh}), $\hat{u_h}$ the solution obtained of (\ref{corr_mult_reh}) such that $\hat{u_h}=\hat{\phi}C_h$ with $\hat{\phi}=\tilde{\phi}+m=u_{ex}+\epsilon P+m$.
			\item With the previous property, we can show that, with $m$ sufficiently large, the error no longer depends on the solution but only on the perturbation $P$ :
			\begin{equation*}
				\left|\left|\frac{\hat{u_{ex}}}{\hat{\phi}}-C_h\right|\right|_{0,\Omega}\le ch^{k+1}\epsilon\left|\left|P''\right|\right|_{0,\Omega}
			\end{equation*}
			\item We can also prove that when m tends to infinity, the solution obtained with multiplication correction on an elevated problem  (\ref{corr_mult_reh}) converges to the solution obtained with correction by adding (\ref{corr_add}).
		\end{enumerate}
	\end{frame}
	
	\subsection{Numerical results}
	
	\begin{frame}{Correction on exact solution}
		We consider the trigonometric solution on the square ($n_{vert}=100$) and
		$$\tilde{\phi}=u_{ex}\in\mathbb{P}_{10}$$
		\textbf{Results with FEM :}
		\begin{center}
			\pgfimage[width=0.7\linewidth]{images/corr/corr_ana/tab_errors_fem.png}
		\end{center}
		\textbf{Results with $\phi$-FEM :}
		\begin{center}
			\pgfimage[width=0.7\linewidth]{images/corr/corr_ana/tab_errors_phifem.png}
		\end{center}
	\end{frame}

	\begin{frame}{Correction on disturbed solution}
		We consider the trigonometric solution on the square ($n_{vert}=100$) and
		\begin{equation*}
			\tilde{\phi}=u_{ex}+\epsilon P\in\mathbb{P}_{10}
		\end{equation*}
		where $u_{ex}$ the exact solution to the problem, $P$ the perturbation and $\epsilon$ a real number (amplitude of $P$).
		
		We will choose to consider $P$ as being of the same form as our exact solution defined by the parameters $(S_p,f_p,\varphi_p)$ with $\varphi_p=0$. \\
		\color{red}TO DO !\color{black}
	\end{frame}

	\begin{frame}{Correction on disturbed solution - Elevated Problem}
		Try the correction by multiplying on the elevated problem with FEM and $\phi$-FEM :
		\begin{center}
			\pgfimage[width=\linewidth]{images/corr/corr_pert/rehaussement/fig_reh.png}
		\end{center}
	\end{frame}

	\begin{frame}{Correction on a $\phi$-FEM solution}
		We consider the trigonometric solution on the square and
		$$\tilde{\phi}=u_{\phi-FEM}\in\mathbb{P}_{2}$$
		where $u_{\phi-FEM}$ is the solution obtained with $\phi$-FEM ($n_{vert}=32$). \\
		We will correct this solution with the different correction methods by using FEM. \\
		We obtain the following correction terms ($\tilde{C}$ for addition and $C$ for multiplication) :
		\begin{center}
			\pgfimage[width=0.9\linewidth]{images/corr/corr_phifem/results_corr.png}
		\end{center}
	\end{frame}

	\begin{frame}{Correction on a $\phi$-FEM solution - Derivatives}
		We compute the first and second derivatives of $\tilde{\phi}$ according to $x$ and $y$.
		\begin{center}
			\pgfimage[width=\linewidth]{images/corr/corr_phifem/derivees.png}
		\end{center}
		We can see that the second derivatives are quite far from the true derivatives.
	\end{frame}

	\begin{frame}[allowframebreaks]{Correction on a FNO prediction}
		We consider the unknown solution on the circle (with $f$ Gaussian) and
		$\tilde{\phi}=u_{FNO}\in\mathbb{P}_{2}$ where $u_{FNO}$ is the FNO prediction ($n_{vert}=32$). \\
		We will apply the different correction methods on the FNO prediction (of a test sample of size $n_{test}=100$). \\
		Training on 4000 epochs (with bs=64 and lr=0.01) :
		\begin{minipage}{0.52\linewidth}
			\centering
			\pgfimage[width=0.9\linewidth]{images/corr/corr_FNO/misfits_f_gaussienne.png}
		\end{minipage}
		\begin{minipage}{0.46\linewidth}
			\centering
			\pgfimage[width=0.7\linewidth]{images/corr/corr_FNO/infos_test_f_gaussienne.png}
		\end{minipage}	
		
		\newpage
		
		\begin{minipage}{0.38\linewidth}
			\centering
			\pgfimage[width=\linewidth]{images/corr/corr_FNO/boxplot_test_f_gaussienne.png}
		\end{minipage} \;
		\begin{minipage}{0.58\linewidth}
			\centering
			\pgfimage[width=\linewidth]{images/corr/corr_FNO/corr_boxplot.png}
		\end{minipage}
	
		\textbf{increase the degree of the $\mathbb{P}_k$ space ?} \\
		interpolation of $\tilde{\phi}$ : decomposition into a series of Legendre polynomials $\Rightarrow$ analytical expression of the solution valid at any point of $\Omega$ $\Rightarrow$ application of the correction in $\mathbb{P}^10$
	\end{frame}

	\begin{frame}[allowframebreaks]{Correction with other networks}
		\textbf{Idea :} using a neural network to predict a single solution at any point in the domain \\
		\quad input data = a collection of $n_{pts}$ 2D points : $\{(x_i,y_i)\}_{i=1,\dots, n_{pts}}$ \\
		\quad output = solution at each of these points : $\{u_i\}_{i=1,\dots,n_{dots}}$ \\
		\quad \quad with $u_i=\phi(x_i,y_i)w_i$ and $w_i=w_\theta(x_i,y_i)$.
		\begin{enumerate}[\ding{217}]
			\item start with Fully-connected Multi-Layer Perceptron (MLP) but bad derivatives
			\item try with a Physics-Informed Neural Networks (PINNs)
			\begin{minipage}{0.48\linewidth}
				model = MLP with 6-layer network ($\{10,20,60,60,20,10\}$)
				\centering
				\pgfimage[width=\linewidth]{images/corr/corr_networks/MLP_schema.png}
			\end{minipage}
			\begin{minipage}{0.48\linewidth}
				Training over 20000 epochs (first half with lr=0.01, second half with lr=0.001) :
				\centering
				\pgfimage[width=\linewidth]{images/corr/corr_networks/PINNs_loss.png}
			\end{minipage}
		\end{enumerate}
		
		\newpage
		
		We consider the trigonometric solution on the square and $\tilde{\phi}=u_{PINNs}\in\mathbb{P}_{10}$ where $u_{PINNs}$ is the PINNs prediction ($n_{vert}=32$) and we have 
		\begin{equation*}
			||u_{ex}-\tilde{\phi}||_{0,\Omega}^{(rel)}=1.93e-3.
		\end{equation*}
		We will correct this solution with the correction by adding (with FEM and $\phi$-FEM).
		
		\begin{minipage}{0.48\linewidth}
			Standard FEM method :
			\centering
			\pgfimage[width=0.6\linewidth]{images/corr/corr_networks/PINNs_FEM_add.png} \\
			$||u_{ex}-\tilde{\phi}C||_{0,\Omega}^{(rel)}=1.13e-4$ \\
			divided by 246.60 (FEM : 2.80e-2)
		\end{minipage}
		\begin{minipage}{0.48\linewidth}
			$\phi$-FEM method :
			\centering
			\pgfimage[width=0.6\linewidth]{images/corr/corr_networks/PINNs_FEM_add.png} \\
			$||u_{ex}-\tilde{\phi}C||_{0,\Omega}^{(rel)}=1.27e-4$ \\
			divided by 151.34 ($\phi$-FEM : 1.92e-2)
			
		\end{minipage}
		
	\end{frame}
	
	\section{Conclusion} %perspectives
	
	\begin{frame}{Conclusion}
		\begin{enumerate}[\ding{217}]
			\item obtain numerical results on analytical solutions $\Rightarrow$ correction methods considered functional and theoretical results confirmed
			\item correction methods on the FNO predictions not satisfactory
			\item try to increase the degree of the solution : 
			\begin{itemize}
				\item Legendre and MLP not satisfactory 
				\item PINNs : reduction of the error made by conventional methods (FEM and $\phi$-FEM) by a factor of around 100 (correction by addition). 
			\end{itemize}
		\end{enumerate}
		\textbf{Perspectives :}
		\begin{enumerate}[\ding{217}]
			\item try adding PINNs to the output of the FNO (add the PINNs as a layer output that would replace the decomposition into a series of polynomials) $\Rightarrow$ solution at any point in the domain 
			\item carry out some documentation work to find more suitable models than the FNO
			\item consider more complex and time-varying geometries (such as 3D organ geometries)
		\end{enumerate}
	\end{frame}

	\section{Bibliography}
	
	\begin{frame}[allowframebreaks]{}
		\printbibliography[heading=none]
	\end{frame}
	
\end{document}
