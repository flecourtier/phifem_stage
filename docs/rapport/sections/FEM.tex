\section{Finite Element Methods (FEMs)}
\graphicspath{{images/FEM}}

In the following, we will consider the Poisson problem with Dirichlet condition (homogeneous or inhomogeneous):

\textbf{Problem :} Find $u : \Omega \rightarrow \mathbb{R}^d$ such that

\begin{equation*}
	\left\{
		\begin{aligned}
			-\Delta u &= f \; &&\text{in } \; \Omega \\
			u&=g \; &&\text{on } \; \partial\Omega
		\end{aligned}
	\right.
\end{equation*}

with $\Delta$ the Laplace operator and $\Omega\subset\mathbb{R}^d$ a lipschitzian bounded open set (and $\partial\Omega$ its boundary).

\textbf{Associated physical model :} Newtonian gravity, Electrostatics, Fluid dynamics...

\subsection{Standard FEM}

In this section, we will present the standard finite element method. We'll start by presenting some general notions of functional analysis, then explain the general principle of FEM. Then we'll give a few more details on the method and finish by describing the application to the Poisson problem (with Dirichlet condition).

\subsubsection{Some notions of functional analysis.}

\modif{AJOUTER : Déf Espace de Hilbert (+ ce qu'il faut pour def ça : Espace de Soboloev ? ... ) + Déf $L^2$,}

\subsubsection{General principle of the method}

Let's consider a domain $\Omega$ whose boundary is denoted $\partial\Omega$. We seek to determine a function $u$ defined on $\Omega$, solution of a partial differential equation (PDE) for given boundary conditions.

The general approach of the finite element method is to write down the variational formulation of this PDE, thus giving us a problem of the following type:

\textbf{Variational Problem :}
\begin{equation*}
	\text{Find } u\in V \text{ such that } a(u,v)=l(v), \;\forall v\in V
\end{equation*}

where $V$ is a Hilbert space, $a$ is a bilinear form and $l$ is a linear form.

To do this, we multiply the PDE by a test function $v\in V$, then integrate over $L^2(\Omega)$.

The idea of FEM is to use Galerkin's method. We then look for an approximate solution $u_h$ in $V_h$, a finite-dimensional space dependent on a positive parameter $h$ such that

\begin{equation*}
	V_h\subset V, \quad \dim V_h = N_h<\infty, \quad \forall h>0
\end{equation*}

The variational problem can then be approached by :

\textbf{Approach Problem :}
\begin{equation*}
	\text{Find } u_h\in V_h \text{ such that } a(u_h,v_h)=l(v_h), \;\forall v_h\in V
\end{equation*}

As $V_h$ is of finite dimension, we can consider a basis $(\varphi_1,\dots,\varphi_{N_h})$ of $V_h$ and thus decompose $u_h$ on this basis as :

\begin{equation}
	\label{decomp1}
	u_h=\sum_{i=1}^{N_h}u_i\varphi_i	
\end{equation}

The approached problem is then rewritten as

\begin{equation*}
	\text{Find } u_1,\dots,u_{N_h} \text{ such that } \sum_{i=1}^{N_h}u_i a(\varphi_i,v_h)=l(v_h), \;\forall v_h\in V 
\end{equation*}

and

\begin{equation*}
	\text{Find } u_1,\dots,u_{N_h} \text{ such that } \sum_{i=1}^{N_h}u_i a(\varphi_i,\varphi_j)=l(\varphi_j), \;\forall j\in \{1,\dots,N_h\}
\end{equation*}

Solving the PDE involves solving the following linear system:
\begin{equation*}
	AU=b
\end{equation*}
with
\begin{equation*}
	A=(a(\varphi_i,\varphi_j))_{1\le i,j\le N_h}, \quad U=(u_i)_{1\le i\le N_h} \quad \text{and} \quad b=(l(\varphi_j))_{1\le j\le N_h}
\end{equation*}

\subsubsection{Some details on FEM}

After having seen the general principle of FEM, it remains to define the $V_h$ spaces and the $\{\varphi_i\}$ basis functions.

\begin{Rem}
	The choice of $V_h$ space is fundamental to have an efficient method that gives a good approximation $u_h$ of $u$. In particular, the choice of the $\{\varphi_i\}$ basis of $V_h$ influences the structure of the $A$ matrix in terms of its sparsity and its condition number.
\end{Rem}

%\modif{Rajouter intro différentes notions abordées !}
%\trad{Pour ce faire, nous aurons besoin de plusieurs notions, qui seront détaillées dans les sections suivantes. Tout d'abord, nous aurons besoin de générer un \textbf{maillage} de notre domaine $\Omega$. Ainsi, nous pourrons résoudre l'EDP de manière discrète, en des points choisis. C'est alors qu'intervient la notion d'\textbf{éléments finis de Lagrange}. Les propriétés de ces éléments, notamment en terme de \textbf{famille affine d'élément fini} est un point clé de la méthode qui va nous permettre de ramener chaque élément du maillage à un \textbf{élément de référence} à l'aide d'une \textbf{transformation géométrique}.}

To do this, we'll need several notions, which will be detailed in the following sections. First, we'll need to generate a \textbf{mesh} of our $\Omega$ domain. This will enable us to solve the PDE discretely at selected points. This is where the notion of \textbf{finite Lagrange elements} comes in. The properties of these elements, particularly in terms of their \textbf{affine family of finite elements}, is a key point of the method, which will enable us to bring each element of the mesh back to a \textbf{reference element} by using a \textbf{geometric transformation}. To describe these steps, we'll need to know 2 basic concepts: the \textbf{\modif{unisolvance}} principle and the definitions of the \textbf{polynomial spaces} used ($\mathbb{P}_k$ and $\mathbb{Q}_k$).


\paragraph{Unisolvance}

\begin{Def}
	Let $\Sigma=\{a_1,\dots,a_N\}$ be a set of $N$ distinct points of $\mathbb{R}^n$. Let $P$ be a finite-dimensional vector space of $\mathbb{R}^n$ functions taking values in $\mathbb{R}$. We say that $\Sigma$ is $P$-\modif{unisolvent} if and only if for all real $\alpha_1,\dots,\alpha_N$, there exists a unique element $p$ of $P$ such that $p(a_i)=\alpha_i,i=1,\dots,N$. 
	This means that the function
	\begin{align*}
		L \; : \; P &\rightarrow \mathbb{R}^N \\
		p &\mapsto(p(a_1),\dots,p(a_N))
	\end{align*}
	is bijective.
\end{Def}

\begin{Rem}
	In practice, to show that $\Sigma$ is $P$-\modif{unisolvent}, we simply check that $\dim P= \text{card } (\Sigma)$ and then prove the injectivity or surjectivity of $L$. The injectivity of $L$ is demonstrated by showing that the only function of $P$ that annuls on all points of $\Sigma$ is the null function. The surjectivity of $L$ is shown by identifying a family $p_1,\dots,p_N$ of elements of $P$ such that $p_i(a_j)=\delta_{ij}$. Given real $\alpha_1,\dots,\alpha_N$, the function $p=\sum_{i=1}^N\alpha_i p_i$ then verifies $p(a_j)=\alpha_j,j=1\dots,N$. 
\end{Rem}

\begin{Rem}
	We call local basis functions of element $K$ the $N$ functions $p_1,\dots,p_N$ of $P$ such that
	\begin{equation*}
		p_i(a_j)=\delta_{ij},\quad 1\le i,j\le N
	\end{equation*}
\end{Rem}

\paragraph{Polynomial space}

Let $\mathbb{P}_k$ be the vector space of polynomials of total degree less than or equal to $k$.

\begin{enumerate}[label=\textbullet]
	\item In $\mathbb{R}$ : $\mathbb{P}_k=\text{Vect}\{1,X,\dots,X^k\}$ and $\dim\mathbb{P}_k=k+1$ 
	\item In $\mathbb{R}^2$ : $\mathbb{P}_k=\text{Vect}\{X^iY^j,0\le i+j\le k\}$ and $\dim\mathbb{P}_k=\frac{(k+1)(k+2)}{2}$
	\item In $\mathbb{R}^3$ : $\mathbb{P}_k=\text{Vect}\{1,X^iY^jZ^l,0\le i+j+l\le k\}$ and $\dim\mathbb{P}_k=\frac{(k+1)(k+2)(k+3)}{6}$
\end{enumerate}

Let $\mathbb{Q}_k$ be the vector space of polynomials of degree less than or equal to $k$ with respect to each variable.

\begin{enumerate}[label=\textbullet]
	\item In $\mathbb{R}$ : $\mathbb{Q}_k=\mathbb{P}_k$. 
	\item In $\mathbb{R}^2$ : $\mathbb{Q}_k=\text{Vect}\{X^iY^j,0\le i,j\le k\}$ and $\dim\mathbb{Q}_k=(k+1)^2$
	\item In $\mathbb{R}^3$ : $\mathbb{Q}_k=\text{Vect}\{1,X^iY^jZ^l,0\le i,j,l\le k\}$ and $\dim\mathbb{Q}_k=(k+1)^3$
\end{enumerate}

\paragraph{Finite Lagrange Element}

The most classic and simplest type of finite element is the Lagrange finite element.

\begin{Def}[Lagrange Finite Element]
	A finite Lagrange element is a triplet $(K,\Sigma,P)$ such that 
	\begin{enumerate}[label=\textbullet]
		\item $K$ is a geometric element of $\mathbb{R}^n$ ($n=1,2$ or $3$), compact, connected and of non-empty interior.
		\item $\Sigma=\{a_1,\dots,a_N\}$ is a finite set of $N$ distinct points of $K$.
		\item $P$ is a finite-dimensional vector space of real functions defined on $K$ and such that $\Sigma$ is $P$-\modif{unisolvent} (so $\dim P=N$).
	\end{enumerate}
\end{Def}

\begin{Example}
	Let $K$ be the segment $[a_1,a_2]$. Let's show that $\Sigma=\{a_1,a_2\}$ is $P$-\modif{unisolvent} for $P=\mathbb{P}^1$. Since $\{1,x\}$ is a base of $\mathbb{P}^1$, we have $\dim P = \text{card } \Sigma = 2$. 
	
	Moreover, we can write $p_i=\alpha_i x+\beta_i, i=1,2$. Thus
	\begin{equation*}
		\left\{\begin{aligned}
			&p_1(a_1)=1 \\
			&p_1(a_2)=0
		\end{aligned}\right. \quad \iff	\quad
		\left\{\begin{aligned}
			&\alpha_1 a_1+\beta_1=1 \\
			&\alpha_1 a_2+\beta_1=0
		\end{aligned}\right. \quad \iff \quad
		\left\{\begin{aligned}
		&\alpha_1 = \frac{1}{a_1-a_2} \\
		&\beta_1 = -\frac{a_2}{a_1-a_2}
	\end{aligned}\right.
	\end{equation*}
	and
	\begin{equation*}
		\left\{\begin{aligned}
			&p_2(a_1)=0 \\
			&p_2(a_2)=1
		\end{aligned}\right. \quad \iff	\quad
		\left\{\begin{aligned}
			&\alpha_2 a_1+\beta_2=0 \\
			&\alpha_2 a_2+\beta_2=1
		\end{aligned}\right. \quad \iff \quad
		\left\{\begin{aligned}
			&\alpha_1 = \frac{1}{a_2-a_1} \\
			&\beta_1 = -\frac{a_1}{a_2-a_1}
		\end{aligned}\right.
	\end{equation*}
	Thus
	\begin{equation*}
		p_1(x)=\frac{x-a_2}{a_1-a_2} \quad \text{and} \quad p_2(x)=\frac{x-a_1}{a_2-a_1}
	\end{equation*}
	We deduce the surjectivity of $L$ and $\Sigma$ is $\mathbb{P}^1$-\modif{unisolvent}. 
	
	Thus $(K,\Sigma,P)$ is a Lagrange Finite Element.
	\end{Example}

	\begin{Def}
		Two finite elements $(\hat{K},\hat{\Sigma},\hat{P})$ and $(K,\Sigma,P)$ are affine-equivalent if and only if there exists an inversible affine function $F$ such that
		\begin{enumerate}[label=\textbullet]
			\item $K=F(\hat{K})$
			\item $a_i=F(\hat{a_i}),i=1,\dots,N$ 
			\item $P=\{\hat{p}\circ F^{-1},\hat{p}\in\hat{P}\}$.
		\end{enumerate}
		We then call an \textbf{affine family of finite elements} a family of finite elements, all affine-equivalent to the same element $(\hat{K},\hat{\Sigma},\hat{P})$, called the \textbf{reference element}.
	\end{Def}

	\begin{Rem}
		Let $(\hat{K},\hat{\Sigma},\hat{P})$ and $(K,\Sigma,P)$ be two affine-equivalent finite elements, via an $F$ transformation. Let $\hat{p_i}$ be the local basis functions on $\hat{K}$. Then the local basis functions on $K$ are $p_i=\hat{p_i}\circ F^{-1}$.
	\end{Rem}
		
	\begin{Rem}
		In practice, working with an affine family of finite elements means that all integral calculations can be reduced to calculations on the reference element. 
		
		The reference elements in 1D, 2D triangular and 3D tetrahedral are :
		\begin{figure}[H]
			\centering
			\includegraphics[width=0.8\linewidth]{"reference_element.png"}
			\captionof{figure}{Example of reference Elements}
		\end{figure}
	\end{Rem}

\paragraph{Mesh}

In 1D, the construction of a mesh consists in creating a subdivision of the interval $[a,b]$. We can extend this definition in 2D and 3D by considering that a mesh is formed by a family of elements (see Fig~\ref{triangle_mesh}):
\begin{equation*}
	\mathcal{T}_h = \left\{K_1,\dots,K_{N_e}\right\}
\end{equation*} 
where $N_e$ is the number of elements. 

In 2D, these elements can be triangles or rectangles. In 3D, they can be tetrahedrons, parallelepipeds or prisms.

\begin{figure}[H]
	\centering
	\includegraphics[width=0.3\linewidth]{"triangle_mesh.png"}
	\captionof{figure}{Example of a triangular mesh on a circles}
	\label{triangle_mesh}
\end{figure}

\paragraph{Geometric transformation}

A mesh is generated by
\begin{enumerate}[label=\textbullet]
	\item A reference element noted $\hat{K}$.
	\item A family of geometric transformations mapping $\hat{K}$ to the elements $\left\{K_1,\dots,K_{N_e}\right\}$. Thus, for a cell $K\in\mathcal{T}_h$, we denote $T_K$ the geometric transformation mapping $\hat{K}$ to $K$ :
	\begin{equation*}
	T_K : \hat{K}\rightarrow K
	\end{equation*}
\end{enumerate}

\begin{figure}[H]
	\centering
	\includegraphics[width=0.4\linewidth]{"geometric_trans.png"}
	\captionof{figure}{Geometric transformation applied to a triangle}
	\label{trans_geo}
\end{figure}

Let $(\hat{K},\hat{\Sigma},\hat{P})$ be the finite reference element with 
\begin{enumerate}[label=\textbullet]
	\item the nodes of the reference element $\hat{K}$ : $\hat{\Sigma}=\{\hat{a}_1,\dots,\hat{a}_n\}$
	\item the local base functions of $\hat{K}$: $\{\hat{\psi}_1,\dots,\hat{\psi}_n\}$ (also called form functions)
\end{enumerate}

So for each $K\in\mathcal{T}_h$, we consider a tuple $\{a_{K,1},\dots,a_{K,n}\}$ (degrees of freedom) and the associated geometric transformation is defined by :
\begin{equation*}
	T_K : \hat{x}\mapsto\sum_{i=1}^{n}a_{K,i}\hat{\psi}_i(\hat{x})
\end{equation*}

In particular, we have
\begin{equation*}
	T_K(\hat{a_i})=a_{K,i}, \quad i=1,\dots,n
\end{equation*}

\begin{Rem}
	In particular, if the form functions are affine, the geometric transformations will be too. This is an interesting property, as the gradient of these geometric transformations will be constant.
\end{Rem}

\textbf{Construction of the basis $(\varphi_i)$ of $V_h$ :}

\modif{TO COMPLETE !}

\begin{Rem}
	In the following, we'll assume that these transformations are $C^1$-diffeomorphisms (i.e. the transformation and its inverse are $C^1$ and bijective).
\end{Rem}

\subsubsection{Application to the Poisson problem}

\modif{2 sous-sections ? -> Théorie -> Pratique}

\modif{Ajouter formulation variationnel Poisson}

\begin{Prop}[Lax-Milgram]
	
	Let $a$ be a continuous, coercive bilinear form on $V$ and $l$ a continuous, linear form on $V$. Then the variational problem has a unique solution $u\in V$. 
	
	Moreover, if the bilinear form is symmetrical, $u$ is a solution to the following minimization problem:
	\begin{equation*}
		J(u)=\min_{v\in V} J(v), \quad J(v)=\frac{1}{2}a(v,v)-l(v)
	\end{equation*}
\end{Prop}

It can then be shown that the Poisson problem with Dirichlet condition has a unique weak solution \modif{$u\in H_0^1(\Omega)$}.

\modif{rajouter preuve}


\modif{Rajouter : Calcul assemblage matrice dans le cas de ce problème ?}

\subsection{$\phi$-FEM}

\modif{TO COMPLETE !}

%The Φ-FEM method has an advantage that is very interesting in the context of organ geometries. Indeed, this type of geometry can deform in time and meshing a fictitious domain around this geometry avoids having to remesh the geometry in time. Thus only the levelset function will be modified and the mesh can be fixed. Moreover, a Cartesian mesh of the fictitious domain allows us to use the same type of neural network as those applied to images: this is the subject that will be approached during the internship.

