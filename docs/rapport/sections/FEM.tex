\section{Finite Element Methods (FEMs)}
\graphicspath{{images/FEM}}

In the following, we will consider the Poisson problem with Dirichlet condition (homogeneous or inhomogeneous):

\textbf{Problem :} Find $u : \Omega \rightarrow \mathbb{R}^d$ such that

\begin{equation*}
	\left\{
		\begin{aligned}
			-\Delta u &= f \; &&\text{in } \; \Omega \\
			u&=g \; &&\text{on } \; \partial\Omega
		\end{aligned}
	\right.
\end{equation*}

with $\Delta$ the Laplace operator and $\Omega\subset\mathbb{R}^d$ a lipschitzian bounded open set (and $\partial\Omega$ its boundary).

\textbf{Associated physical model :} Newtonian gravity, Electrostatics, Fluid dynamics...

\subsection{Standard FEM}

\subsubsection{Some notions of functional analysis.}

\modif{AJOUTER : Déf Espace de Hilbert (+ ce qu'il faut pour def ça : Espace de Soboloev ? ... ) + Déf $L^2$,}

\subsubsection{General principle of the method}

Let's consider a domain $\Omega$ whose boundary is denoted $\partial\Omega$. We seek to determine a function $u$ defined on $\Omega$, solution of a partial differential equation (PDE) for given boundary conditions.

The general approach of the finite element method is to write down the variational formulation of this PDE, thus giving us a problem of the following type:

\textbf{Variational Problem :}
\begin{equation*}
	\text{Find } u\in V \text{ such that } a(u,v)=l(v), \;\forall v\in V
\end{equation*}

where $V$ is a Hilbert space, $a$ is a bilinear form and $l$ is a linear form.

To do this, we multiply the PDE by a test function $v\in V$, then integrate over $L^2(\Omega)$.

The idea of FEM is to use Galerkin's method. We then look for an approximate solution $u_h$ in $V_h$, a finite-dimensional space dependent on a positive parameter $h$ such that

\begin{equation*}
	V_h\subset V, \quad \dim V_h = N_h<\infty, \quad \forall h>0
\end{equation*}

The variational problem can then be approached by :

\textbf{Approach Problem :}
\begin{equation*}
	\text{Find } u_h\in V_h \text{ such that } a(u_h,v_h)=l(v_h), \;\forall v_h\in V
\end{equation*}

As $V_h$ is of finite dimension, we can consider a basis $(\varphi_1,\dots,\varphi_{N_h})$ of $V_h$ and thus decompose $u_h$ on this basis as :

\begin{equation}
	\label{decomp1}
	u_h=\sum_{i=1}^{N_h}u_i\varphi_i	
\end{equation}

The approached problem is then rewritten as

\begin{equation*}
	\text{Find } u_1,\dots,u_{N_h} \text{ such that } \sum_{i=1}^{N_h}u_i a(\varphi_i,v_h)=l(v_h), \;\forall v_h\in V 
\end{equation*}

and

\begin{equation*}
	\text{Find } u_1,\dots,u_{N_h} \text{ such that } \sum_{i=1}^{N_h}u_i a(\varphi_i,\varphi_j)=l(\varphi_j), \;\forall j\in \{1,\dots,N_h\}
\end{equation*}

Solving the PDE involves solving the following linear system:
\begin{equation*}
	AU=b
\end{equation*}
with
\begin{equation*}
	A=(a(\varphi_i,\varphi_j))_{1\le i,j\le N_h}, \quad U=(u_i)_{1\le i\le N_h} \quad \text{and} \quad b=(l(\varphi_j))_{1\le j\le N_h}
\end{equation*}

\subsubsection{Some details on FEM}

\modif{Notions à aborder : ef de Lagrange + unisolvance + maillage + Transformation géométrique}

After having seen the general principle of FEM, it remains to define the $V_h$ spaces and the $\{\varphi_i\}$ basis functions.

\begin{Rem}
	The choice of $V_h$ space is fundamental to have an efficient method that gives a good approximation $u_h$ of $u$. In particular, the choice of the $\{\varphi_i\}$ basis of $V_h$ influences the structure of the $A$ matrix in terms of its sparsity and its condition number.
\end{Rem}

\paragraph{Finite Lagrange Element}

\trad{Le type le plus classique et le plus simple d'éléments finis sont les éléments finis de Lagrange.}

\begin{Def}[Lagrange Finite Element]
	\trad{Un élément fini de Lagrange est un triplet $(K,\Sigma,P)$ tel que 
	\begin{enumerate}[label=\textbullet]
		\item $K$ est un élément géométrique de $\mathbb{R}^n$ ($n=1,2$ ou $3$), compact, connexe et d'intérieur non vide.
		\item $\Sigma=\{a_1,\dots,a_N\}$ est un ensemble fini de $N$ points distincts de $K$.
		\item $P$ est un espace vectoriel de dimension finie de fonctions réelles définies sur $K$ et tel que $\Sigma$ soit $P$-unisolvant (donc $\dim P=N$).
	\end{enumerate}}
\end{Def}

\begin{Rem}
	\trad{On dit que $\Sigma$ est $P$-unisolvant si et seulement si pour tous réels $\alpha_1,\dots,\alpha_N$, il existe un unique élément $p$ de $P$ tel que $p(a_i)=\alpha_i,i=1,\dots,N$. 
	Ceci revient à dire que la fonction}
	\begin{align*}
		L \; : \; P &\rightarrow \mathbb{R}^N \\
		p &\mapsto(p(a_1),\dots,p(a_N))
	\end{align*}
	\trad{est bijective.}
\end{Rem}

\begin{Rem}
	En pratique, pour montrer que $\Sigma$ est $P$-unisolvant, on vérifiera simplement que $\dim P=card(\Sigma)$ puis on montrera l'injectivité ou la surjectivité de $L$. L'injectivité  de $L$ se démontre en établissant que la seule fonction de $P$ s'annulant sur tous les points de $\Sigma$ est la fonction nulle. La surjectivité de $L$ se démontre en exhibant une famille $p_1,\dots,p_N$ d'éléments de $P$ tels que $p_i(a_j)=\delta_{ij}$. En effet, étant donné des réels $\alpha_1,\dots,\alpha_N$, la fonction $p=\sum_{i=1}^N\alpha_i p_i$ vérifie alors $p(a_j)=\alpha_j,j=1\dots,N$. 
\end{Rem}

\modif{A mettre plus loin (base de $P_1$ à définir d'abord)}
\begin{Example}
	\trad{Soit $K$ le segment $[a_1,a_2]$. Montrons que $\Sigma=\{a_1,a_2\}$ est $P$-unisolvant pour $P=\mathbb{P}^1$. Comme $\{1,x\}$ est une base de $\mathbb{P}^1$, on a bien $\dim P = \text{card } \Sigma = 2$. 
		
	De plus, on peut écrire $p_i=\alpha_i x+\beta_i, i=1,2$. Ainsi
	\begin{equation*}
		\left\{\begin{aligned}
			&p_1(a_1)=1 \\
			&p_1(a_2)=0
		\end{aligned}\right. \quad \iff	\quad
		\left\{\begin{aligned}
			&\alpha_1 a_1+\beta_1=1 \\
			&\alpha_1 a_2+\beta_1=0
		\end{aligned}\right. \quad \iff \quad
		\left\{\begin{aligned}
		&\alpha_1 = \frac{1}{a_1-a_2} \\
		&\beta_1 = -\frac{a_2}{a_1-a_2}
	\end{aligned}\right.
	\end{equation*}
	et
	\begin{equation*}
		\left\{\begin{aligned}
			&p_2(a_1)=0 \\
			&p_2(a_2)=1
		\end{aligned}\right. \quad \iff	\quad
		\left\{\begin{aligned}
			&\alpha_2 a_1+\beta_2=0 \\
			&\alpha_2 a_2+\beta_2=1
		\end{aligned}\right. \quad \iff \quad
		\left\{\begin{aligned}
			&\alpha_1 = \frac{1}{a_2-a_1} \\
			&\beta_1 = -\frac{a_1}{a_2-a_1}
		\end{aligned}\right.
	\end{equation*}
	Donc
	\begin{equation*}
		p_1(x)=\frac{x-a_2}{a_1-a_2} \quad \text{and} \quad p_2(x)=\frac{x-a_1}{a_2-a_1}
	\end{equation*}
	On en déduit la surjectivité de $L$ et $\Sigma$ est $\mathbb{P}^1$-unisolvant.}
	\end{Example}

\paragraph{Mesh}

\trad{La première étape consiste à construire un maillage de $\Omega$.}

\begin{Rem}
	\trad{La partie de génération de maillage est une étape importante }
\end{Rem}

\subsubsection{Application to the Poisson problem}

\modif{Ajouter formulation variationnel Poisson}

\begin{Prop}[Lax-Milgram]
	
	Let $a$ be a continuous, coercive bilinear form on $V$ and $l$ a continuous, linear form on $V$. Then the variational problem has a unique solution $u\in V$. 
	
	Moreover, if the bilinear form is symmetrical, $u$ is a solution to the following minimization problem:
	\begin{equation*}
		J(u)=\min_{v\in V} J(v), \quad J(v)=\frac{1}{2}a(v,v)-l(v)
	\end{equation*}
\end{Prop}

It can then be shown that the Poisson problem with Dirichlet condition has a unique weak solution.

\modif{rajouter preuve}

\subsection{$\phi$-FEM}

%The Φ-FEM method has an advantage that is very interesting in the context of organ geometries. Indeed, this type of geometry can deform in time and meshing a fictitious domain around this geometry avoids having to remesh the geometry in time. Thus only the levelset function will be modified and the mesh can be fixed. Moreover, a Cartesian mesh of the fictitious domain allows us to use the same type of neural network as those applied to images: this is the subject that will be approached during the internship.

