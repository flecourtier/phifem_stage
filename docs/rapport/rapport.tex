%compile then compile with biber (for the biblio) then recompile
\documentclass[12pt]{article}

\usepackage{a4wide} % increase the typeset area
\usepackage{bm}
\usepackage{amsmath,amssymb}
\usepackage{enumitem}
\usepackage{graphicx}
\usepackage{color}
\usepackage{float} %to place figure H
\usepackage{multirow} %multirow for table
%\usepackage[super]{natbib} %exponant biblio
\usepackage{fourier} %for danger sign
\usepackage{chngcntr}
\usepackage{pifont} %for more symbol in enumerate

% Useful packages for table

\usepackage{array,multirow,makecell}
%\usepackage[linesnumbered]{algorithm2e}
\usepackage[table]{xcolor}
\setcellgapes{1pt}
\makegapedcells
\newcolumntype{R}[1]{>{\raggedleft\arraybackslash }b{#1}}
\newcolumntype{L}[1]{>{\raggedright\arraybackslash }b{#1}}
\newcolumntype{C}[1]{>{\centering\arraybackslash }b{#1}}



%hyperref
\usepackage[colorlinks]{hyperref}
\hypersetup{
	colorlinks=true,
	linkcolor=blue,
	filecolor=magenta,      
	urlcolor=blue,
	citecolor=blue
}

% captions
\usepackage{caption}
\newcommand{\vect}[1]{\hat{\boldsymbol{#1}}}
\usepackage{subcaption}
\counterwithin{figure}{section}
\makeatletter
\usepackage[labelformat=simple]{subcaption}
\newcommand\captionsubfigure{%
	\renewcommand\p@subfigure{}
	\renewcommand\thesubfigure{\thefigure.\alph{subfigure}}
}
\makeatother

%to make the appendix
\usepackage{appendix}

%code
\usepackage{listings}
\definecolor{backcolor}{RGB}{240, 240, 240}
\lstdefinestyle{bash}{
	commentstyle=\color{green},
	morecomment=[l][\color{magenta}]{\#},
	backgroundcolor=\color{backcolor},  
	breakatwhitespace=false,
	keepspaces=true,                        
	showspaces=false,                
	showstringspaces=false,
	showtabs=false,                  
	tabsize=1    
}

%for footnote
\usepackage[symbol]{footmisc}
\renewcommand{\thefootnote}{\fnsymbol{footnote}}

%bibliography (with section)
%\usepackage[backend=biber,style=numeric,sorting=nyt]{biblatex}


%\usepackage{biblatex} 
%\addbibresource{biblio.bib}


% Titlepage
\newcommand{\reporttitle}{Internship Report : Innovative non-conformal finite element methods for augmented surgery}
\newcommand{\reportauthorOne}{Frédérique Lecourtier}
\newcommand{\reportsupervisorOne}{Michel Duprez}
\newcommand{\reportsupervisorTwo}{Emmanuel Franck}
\newcommand{\reportsupervisorThree}{Vanessa Lleras}
\newcommand{\reporttype}{Coursework}

%Algorithm
\usepackage{xcolor}
\usepackage[linesnumbered,noline,boxed,commentsnumbered]{algorithm2e}
%\usepackage[noline, linesnumbered]{algorithm2e}% http://ctan.org/pkg/algorithm2e
\SetNlSty{bfseries}{\color{black}}{}

% pour rajouter des commentaires en rouge !
\newcommand{\tmcolor}[2]{{\color{#1}{#2}}}
\newcommand{\modif}[1]{\tmcolor{red}{#1}}
\newcommand{\trad}[1]{\tmcolor{blue}{#1}}

\usepackage{amsthm}
\newtheorem*{Rem}{\textit{Remark}}
\newtheorem{Prop}{Proposition}[section]
\newtheorem{Def}{Definition}[section]
\newtheorem*{Example}{Example}

\setlength\parindent{0pt}

\usepackage{titlesec}
\setcounter{secnumdepth}{4}
\titleformat{\paragraph}
{\normalfont\normalsize\bfseries}{\theparagraph}{1em}{}
\titlespacing*{\paragraph}
{0pt}{3.25ex plus 1ex minus .2ex}{1.5ex plus .2ex}

\usepackage{fontawesome5}

\begin{document}
	\nocite{*}
	
	\begin{titlepage}

\newcommand{\HRule}{\rule{\linewidth}{0.5mm}}

\begin{center}
	\includegraphics[width = 0.5\linewidth]{images/logo-mimesis.png} \\ [1.5cm] 

	\textsc{\Large University of Strasbourg}\\[0.5cm] 
	\textsc{\large Master CSMI}\\[0.95cm] 
	
	\HRule \\[0.4cm]
	\huge\bfseries\reporttitle\par % Title of your document
	\HRule \\[0.4cm]
\end{center}

\vspace{1cm}

\begin{flushleft} \large
	\begin{minipage}{0.4\hsize}
		\textit{Authors:}\\
		\reportauthorOne
	\end{minipage} \hfill 
	\begin{minipage}{0.4\hsize}
		\textit{Supervisors:}\\
		\reportsupervisorOne\\
		\reportsupervisorTwo
	\end{minipage}
\end{flushleft}
\vspace{3 cm}
\makeatletter
Date: \@date
\hfill
\includegraphics[width = 0.2\linewidth]{images/inria.png}\\[1.5cm] 

\vfill % Fill the rest of the page with whitespace



\makeatother


\end{titlepage}


	\tableofcontents
	
	\newpage
	
%	\textbf{PLAN général}
%	\begin{enumerate}
%		\item \textbf{Intro :} Ce stage est un stage de M2, dans le cadre du master CSMI. C'est la suite d'un projet effectué pendant le premier semestre.... explication rapide du projet (colab entre Cemosis et Mimesis dans le but de blablabla)
%		\begin{enumerate}
%			\item \textbf{Présentation de Mimesis}
%			\item \textbf{Contexte : }Mimesis est spécialisé dans la simulation en temps réel de blabla. C'est pourquoi ils ont développé une méthode appelée PhiFEM (explication des intérêts de PhiFEM).
%			\item \textbf{Objectifs :} On cherche ici à améliorer la précision de la solution ainsi que les temps pour l'obtenir en passant par le biais d'un FNO que l'on va entraîner avec des solutions PhiFEM qui sont bien adapté à ce type de réseau de neurones, due aux grilles cartésienne.
%		\end{enumerate}
%		\item \textbf{Finite Element Methods (FEM)}
%		\begin{enumerate}
%			\item \textbf{Standard FEM}
%			\item \textbf{PhiFEM}
%		\end{enumerate}
%		\item \textbf{Fourier Neural Operator (FNO)}
%		\item \textbf{Correction :} \\
%		Présentation des différentes méthodes de Correction (2 cas tests : solution analytique, sol FNO) / Legendre / Modèle Dense ?  \\
%		On veut mettre un schéma qui explique Entrainement du FNO -> Sortie -> Correction (-> Legendre ?)
%		\item \textbf{Résultats ?}
%		\item \textbf{Conclu}
%		\item \textbf{Bibliography}
%		\item \textbf{Appendix} : Organisation of the repository +  Documentation + Github actions
%	\end{enumerate}

	\newpage
	\section{Introduction}

This end of study internship is a 2nd year internship in the CSMI Master ("Calcul Scientifique et Mathématiques de l'Information") of the University of Strasbourg. It is the continuation of a project done during the first semester of M2, the main objective of this project was the discovery of an innovative non-conformal finite element method for augmented surgery, the $\phi$-FEM method. The purpose of this project was to have Cemosis and Mimesis collaborate through the use of Feel++ software (developed by Cemosis) in the framework of the $\phi$-FEM method (one of the research topics of the Mimesis team). This project was followed by a 6-month internship whose main objective was to correct the output of a Fourier Neural Operator (FNO) by a solver using the $\phi$-FEM method.

\subsection{Scientific Context}

Finite element methods (FEM) are used to solve partial differential equations numerically. These can, for example, represent analytically the dynamic behavior of certain physical systems (mechanical, thermodynamic, acoustic, etc.). Among other things, it is a discrete algorithm for determining the approximate solution of a partial differential equation (PDE) on a compact domain with boundary conditions. 

The standard FEM method, which requires precise meshing of the domain under consideration and, in particular, fitting with its boundary, has its limitations. In particular, in the medical field, meshing complex and evolving geometries such as organs (e.g. the liver) can be very costly. More specifically, in the application context of creating real-time digital twins of an organ, the standard FEM method would require complete remeshing of the organ each time it is deformed, which in practice is not workable. 

This is why other methods, known as non-conformal finite element methods, have emerged in the last few years. These include CutFEM \cite{burman_cutfem_2015} or XFEM \cite{moes_x-fem_2002}, based on the idea of introducing a fictitious domain larger than the domain under consideration. We're interested here in another non-conformal method, which we'll present in more detail later, called $\phi$-FEM. We'll only use it in the context of Poisson problem solving, for Dirichlet boundary conditions \cite{duprez_phi-fem_2020}. But the method has been extended to Neumann conditions \cite{duprez_new_2023} and then to solve various mechanical problems, including linear elasticity \cite[Chapter~2]{cotin_phi-fem_nodate} and heat transfer problems \cite[Chapter~5]{cotin_phi-fem_nodate}.

\subsection{Presentation of the team}

Created in January 2021 within ICube laboratory at the University of Strasbourg, MLMS\footnote{MLMS : \url{https://mlms.icube.unistra.fr/en/index.php/Presentation}} ("Machine Learning, Modélisation et Simulation") team is interested in data, models and simulations for medical science and human motion. It brings together computer scientists, mathematicians, bio-mechanicians, and neuroscientists to develop functional, physical, and geometric models around a transverse axis "Assistance to medical interventions by computer". MLMS hosts the MIMESIS\footnote{MIMESIS : \url{https://mimesis.inria.fr/}} project-team as a sub-team. The MIMESIS research team aims at creating real-time digital twins of an organ, with main application domains as surgical training and surgical guidance during complex interventions. In 2023, a new inria team NECTARINE will be created within MLMS, who will focus on scientific challenges related to neuro-stimulation in the clinical context. 

MIMESIS, directed by Stéphane Cotin, is a joint Inria\footnote{Inria : \url{https://www.inria.fr/fr}} ("Institut national de recherche en sciences et technologies du numérique") and CNRS\footnote{CNRS : \url{https://www.cnrs.fr/fr}} ("Centre national de la recherche scientifique") Research Team. The Mimesis research team is working on a set of scientific challenges in scientific computing, data assimilation, machine learning and control, with the goal of creating real-time digital twins of an organ.

\subsection{Objectives}

The main objective of the internship was to combine finite element methods and Machine Learning in order to solve the Poisson problem with Dirichlet condition. More precisely, we want to train a neural network called Fourier Neural Network (FNO) \cite{li_fourier_2021} to predict the solutions of a PDE for a given problem family (i.e. a "type" of source term). This neural network is trained with a data set consisting of the $\phi$-FEM solutions of the problems considered. The predictions of this neural network will then be fed back into a finite element solver to apply a correction to improve the accuracy of the solution : this was the subject covered during the internship. The finite element methods considered will be presented in Section \ref{FEMs} and the FNO in Section \ref{FNO}.

It is important to note that the $\phi$-FEM method has an advantage that is very interesting in the context of organ geometries. Indeed, this type of geometry can deform in time and meshing a fictitious domain around this geometry avoids having to remesh the geometry in time. Thus only the levelset function will be modified and the mesh can be fixed. Moreover, a Cartesian mesh of the fictitious domain allows us to use the same type of neural network as those applied to images (especially FNO).

To be more precise, we will test different correction methods (presented in Section \ref{Corr.methods}) on different problems (presented in Section \ref{Corr.problems}) which will enable us to use the network prediction to help the solver get as close as possible to the solution. We will start by testing these different types of solver on an analytical solution (Section \ref{Corr.results.ana}), then on a "manually perturbed" solution (Section \ref{Corr.results.disturbed}) and finally on a $\phi$-FEM solution (Section \ref{Corr.results.phifem}).

After testing the various types of correction on the previous test cases, we'll apply these same methods to the prediction of an FNO (Section \ref{Corr.results.FNO}). The main objective is to enable the combination of FNO and correction to be more accurate than the conventional $\phi$-FEM solver. By first testing the different corrections on the previous test cases, we hope to get an idea of the order of errors to be expected. During the course of the internship, we realized that the results obtained on the FNO did not correspond to the expected analytical results. For this reason, other types of neural networks were considered, namely multi-perceptron networks (Section \ref{Corr.results.neural_net.MLP}) and PINNs (Section \ref{Corr.results.neural_net.PINNs}), with the aim of checking whether the results obtained are related to the use of the FNO.

\subsection{Deliverables}

In the context of the internship, the following deliverables are provided:

\begin{enumerate}[label=\textbullet]
	\item a \href{https://github.com/flecourtier/phifem_stage/blob/main/docs/suivi/suivi.pdf}{weekly tracking report}, written in French, was produced as the internship progressed, listing the objectives and results for each week.
	\item a \href{https://github.com/flecourtier/phifem_stage}{github repository} containing all the code allowing to reproduce the results presented in this report, as well as the documents written during the internship. The codes have been implemented in Python: for the finite element solvers, we'll be using the FEniCS library, and for the neural network implementation, we'll be using Tensorflow and Pytorch.
	\item an \href{https://flecourtier.github.io/phifem_stage/phifem_project/1.0.3/main_page.html}{online report} generated with a tool called antora\footnote{Antora : \url{https://antora.org/}}. A continuous integration has been set up on github to execute a python code for each new push, enabling the latex file to be converted directly into this antora documentation.
%	\item a code documentation has also been set up with sphinx\footnote{Sphinx : \url{https://www.sphinx-doc.org/en/master/}}.
\end{enumerate}
	
	\newpage
	\section{Finite Element Methods (FEMs)}
\graphicspath{{images/FEM}}

In the following, we will consider the Poisson problem with Dirichlet condition (homogeneous or inhomogeneous):

\textbf{Problem :} Find $u : \Omega \rightarrow \mathbb{R}^d$ such that

\begin{equation*}
	\left\{
		\begin{aligned}
			-\Delta u &= f \; &&\text{in } \; \Omega \\
			u&=g \; &&\text{on } \; \partial\Omega
		\end{aligned}
	\right.
\end{equation*}

with $\Delta$ the Laplace operator and $\Omega\subset\mathbb{R}^d$ a lipschitzian bounded open set (and $\partial\Omega$ its boundary).

\textbf{Associated physical model :} Newtonian gravity, Electrostatics, Fluid dynamics...

\subsection{Standard FEM}

\subsubsection{Some notions of functional analysis.}

\modif{AJOUTER : Déf Espace de Hilbert (+ ce qu'il faut pour def ça : Espace de Soboloev ? ... ) + Déf $L^2$,}

\subsubsection{General principle of the method}

Let's consider a domain $\Omega$ whose boundary is denoted $\partial\Omega$. We seek to determine a function $u$ defined on $\Omega$, solution of a partial differential equation (PDE) for given boundary conditions.

The general approach of the finite element method is to write down the variational formulation of this PDE, thus giving us a problem of the following type:

\textbf{Variational Problem :}
\begin{equation*}
	\text{Find } u\in V \text{ such that } a(u,v)=l(v), \;\forall v\in V
\end{equation*}

where $V$ is a Hilbert space, $a$ is a bilinear form and $l$ is a linear form.

To do this, we multiply the PDE by a test function $v\in V$, then integrate over $L^2(\Omega)$.

The idea of FEM is to use Galerkin's method. We then look for an approximate solution $u_h$ in $V_h$, a finite-dimensional space dependent on a positive parameter $h$ such that

\begin{equation*}
	V_h\subset V, \quad \dim V_h = N_h<\infty, \quad \forall h>0
\end{equation*}

The variational problem can then be approached by :

\textbf{Approach Problem :}
\begin{equation*}
	\text{Find } u_h\in V_h \text{ such that } a(u_h,v_h)=l(v_h), \;\forall v_h\in V
\end{equation*}

As $V_h$ is of finite dimension, we can consider a basis $(\varphi_1,\dots,\varphi_{N_h})$ of $V_h$ and thus decompose $u_h$ on this basis as :

\begin{equation}
	\label{decomp1}
	u_h=\sum_{i=1}^{N_h}u_i\varphi_i	
\end{equation}

The approached problem is then rewritten as

\begin{equation*}
	\text{Find } u_1,\dots,u_{N_h} \text{ such that } \sum_{i=1}^{N_h}u_i a(\varphi_i,v_h)=l(v_h), \;\forall v_h\in V 
\end{equation*}

and

\begin{equation*}
	\text{Find } u_1,\dots,u_{N_h} \text{ such that } \sum_{i=1}^{N_h}u_i a(\varphi_i,\varphi_j)=l(\varphi_j), \;\forall j\in \{1,\dots,N_h\}
\end{equation*}

Solving the PDE involves solving the following linear system:
\begin{equation*}
	AU=b
\end{equation*}
with
\begin{equation*}
	A=(a(\varphi_i,\varphi_j))_{1\le i,j\le N_h}, \quad U=(u_i)_{1\le i\le N_h} \quad \text{and} \quad b=(l(\varphi_j))_{1\le j\le N_h}
\end{equation*}

\subsubsection{Some details on FEM}

\modif{Notions à aborder : ef de Lagrange + unisolvance + maillage + Transformation géométrique}

After having seen the general principle of FEM, it remains to define the $V_h$ spaces and the $\{\varphi_i\}$ basis functions.

\begin{Rem}
	The choice of $V_h$ space is fundamental to have an efficient method that gives a good approximation $u_h$ of $u$. In particular, the choice of the $\{\varphi_i\}$ basis of $V_h$ influences the structure of the $A$ matrix in terms of its sparsity and its condition number.
\end{Rem}

\paragraph{Finite Lagrange Element}

\trad{Le type le plus classique et le plus simple d'éléments finis sont les éléments finis de Lagrange.}

\begin{Def}[Lagrange Finite Element]
	\trad{Un élément fini de Lagrange est un triplet $(K,\Sigma,P)$ tel que 
	\begin{enumerate}[label=\textbullet]
		\item $K$ est un élément géométrique de $\mathbb{R}^n$ ($n=1,2$ ou $3$), compact, connexe et d'intérieur non vide.
		\item $\Sigma=\{a_1,\dots,a_N\}$ est un ensemble fini de $N$ points distincts de $K$.
		\item $P$ est un espace vectoriel de dimension finie de fonctions réelles définies sur $K$ et tel que $\Sigma$ soit $P$-unisolvant (donc $\dim P=N$).
	\end{enumerate}}
\end{Def}

\begin{Rem}
	\trad{On dit que $\Sigma$ est $P$-unisolvant si et seulement si pour tous réels $\alpha_1,\dots,\alpha_N$, il existe un unique élément $p$ de $P$ tel que $p(a_i)=\alpha_i,i=1,\dots,N$. 
	Ceci revient à dire que la fonction}
	\begin{align*}
		L \; : \; P &\rightarrow \mathbb{R}^N \\
		p &\mapsto(p(a_1),\dots,p(a_N))
	\end{align*}
	\trad{est bijective.}
\end{Rem}

\begin{Rem}
	En pratique, pour montrer que $\Sigma$ est $P$-unisolvant, on vérifiera simplement que $\dim P=card(\Sigma)$ puis on montrera l'injectivité ou la surjectivité de $L$. L'injectivité  de $L$ se démontre en établissant que la seule fonction de $P$ s'annulant sur tous les points de $\Sigma$ est la fonction nulle. La surjectivité de $L$ se démontre en exhibant une famille $p_1,\dots,p_N$ d'éléments de $P$ tels que $p_i(a_j)=\delta_{ij}$. En effet, étant donné des réels $\alpha_1,\dots,\alpha_N$, la fonction $p=\sum_{i=1}^N\alpha_i p_i$ vérifie alors $p(a_j)=\alpha_j,j=1\dots,N$. 
\end{Rem}

\modif{A mettre plus loin (base de $P_1$ à définir d'abord)}
\begin{Example}
	\trad{Soit $K$ le segment $[a_1,a_2]$. Montrons que $\Sigma=\{a_1,a_2\}$ est $P$-unisolvant pour $P=\mathbb{P}^1$. Comme $\{1,x\}$ est une base de $\mathbb{P}^1$, on a bien $\dim P = \text{card } \Sigma = 2$. 
		
	De plus, on peut écrire $p_i=\alpha_i x+\beta_i, i=1,2$. Ainsi
	\begin{equation*}
		\left\{\begin{aligned}
			&p_1(a_1)=1 \\
			&p_1(a_2)=0
		\end{aligned}\right. \quad \iff	\quad
		\left\{\begin{aligned}
			&\alpha_1 a_1+\beta_1=1 \\
			&\alpha_1 a_2+\beta_1=0
		\end{aligned}\right. \quad \iff \quad
		\left\{\begin{aligned}
		&\alpha_1 = \frac{1}{a_1-a_2} \\
		&\beta_1 = -\frac{a_2}{a_1-a_2}
	\end{aligned}\right.
	\end{equation*}
	et
	\begin{equation*}
		\left\{\begin{aligned}
			&p_2(a_1)=0 \\
			&p_2(a_2)=1
		\end{aligned}\right. \quad \iff	\quad
		\left\{\begin{aligned}
			&\alpha_2 a_1+\beta_2=0 \\
			&\alpha_2 a_2+\beta_2=1
		\end{aligned}\right. \quad \iff \quad
		\left\{\begin{aligned}
			&\alpha_1 = \frac{1}{a_2-a_1} \\
			&\beta_1 = -\frac{a_1}{a_2-a_1}
		\end{aligned}\right.
	\end{equation*}
	Donc
	\begin{equation*}
		p_1(x)=\frac{x-a_2}{a_1-a_2} \quad \text{and} \quad p_2(x)=\frac{x-a_1}{a_2-a_1}
	\end{equation*}
	On en déduit la surjectivité de $L$ et $\Sigma$ est $\mathbb{P}^1$-unisolvant.}
	\end{Example}

\paragraph{Mesh}

\trad{La première étape consiste à construire un maillage de $\Omega$.}

\begin{Rem}
	\trad{La partie de génération de maillage est une étape importante }
\end{Rem}

\subsubsection{Application to the Poisson problem}

\modif{Ajouter formulation variationnel Poisson}

\begin{Prop}[Lax-Milgram]
	
	Let $a$ be a continuous, coercive bilinear form on $V$ and $l$ a continuous, linear form on $V$. Then the variational problem has a unique solution $u\in V$. 
	
	Moreover, if the bilinear form is symmetrical, $u$ is a solution to the following minimization problem:
	\begin{equation*}
		J(u)=\min_{v\in V} J(v), \quad J(v)=\frac{1}{2}a(v,v)-l(v)
	\end{equation*}
\end{Prop}

It can then be shown that the Poisson problem with Dirichlet condition has a unique weak solution.

\modif{rajouter preuve}

\subsection{$\phi$-FEM}

%The Φ-FEM method has an advantage that is very interesting in the context of organ geometries. Indeed, this type of geometry can deform in time and meshing a fictitious domain around this geometry avoids having to remesh the geometry in time. Thus only the levelset function will be modified and the mesh can be fixed. Moreover, a Cartesian mesh of the fictitious domain allows us to use the same type of neural network as those applied to images: this is the subject that will be approached during the internship.


	
	\newpage	
	\section{Fourier Neural Operator (FNO)} \label{FNO}
\graphicspath{{images/fourier}}


We will now introduce Fourier Neural Operators (FNO). For more information, please refer to the following articles \modif{ADD REF !}.

%\trad{En traitement d'image, on appelle image des tenseurs de taille $ni\times nj\times nk$ où $ni\times nj$ correspond à la résolution de l'image et $nk$ correspond à son nombre de channels. Par exemple, une image RGB (Red Green Blue) possède $nk=3$ channels. 
%On choisit ici de présenter le FNO comme un opérateur agissant sur des images discrètes. Les articles de référence le présente sous son aspect continu qui est un point de vue intéressant. En effet, c'est grâce à cette propriété que l'on peut l'entraîner/évaluer avec des images de différentes résolutions.}

In image treatment, we call image tensors of size $ni\times nj\times nk$, where $ni\times nj$ corresponds to the image resolution and $nk$ corresponds to its number of channels. For example, an RGB (Red Green Blue) image has $nk=3$ channels. 
We choose here to present the FNO as an operator acting on discrete images. Reference articles present it in its continuous aspect, which is an interesting point of view. Indeed, it is thanks to this property that it can be trained/evaluated with images of different resolutions.

The FNO methodology creates a relationship between two spaces from a finite collection of observed input-output pairs. \modif{Est-ce que je gardes cette phrase ?}

\modif{Section ?}

\subsection{Architecture of the FNO}

%\trad{La figure suivante \ref{FNO} décrit précisément l'architecture du FNO :}

The following figure (Figure \ref{FNO_schema}) describes the FNO architecture in detail:

\begin{figure}[H]
	\includegraphics[width=\linewidth]{"fno_schema.png"}
	\captionof{figure}{Architecture of the FNO}
	\label{FNO_schema}
\end{figure}

%\trad{La structure du FNO est alors la suivante :}

The structure of the FNO is as follows:

\begin{equation*}
	G_\theta = P \circ \mathcal{H}_\theta^1 \circ \dots \circ \mathcal{H}_\theta^L \circ Q
\end{equation*}

\subsubsection{General structure of the FNO} \label{FNO.general}

%\trad{On va maintenant décrire un peu plus en détail la composition de ce schéma.
%\begin{enumerate}[label=\textbullet]
%	\item On part d'input X de shape (batch\_size, height, width, nb\_channels) avec batch\_size le nombre d'images que l'on traites en même temps, height et width les dimensions des images et nb\_channels le nombre de channels. On simplifiera par (bs,ni,nj,nk).
%	\item On effectue une transformation $P$ dans le but de passer à un espace avec plus de channels. Cette étape permet au réseau de construire une représentation suffisement riche des données.  On pourra par exemple effectuer une couche  Dense (aussi appelée fully-connected). 	
%	\item On applique ensuite $L$ couches de Fourier, notées $\mathcal{H}_\theta^l,\; l=1,\dots,L$, dont on détaillera les spécifications dans la section \modif{AJOUTER SECTION}.
%	\item On se ramène alors à la dimension cible en effectuant une transformation $Q$. Dans notre cas le nombre de channels en sortie est de 1.
%	\item On récupère alors la sortie du modèle $Y$ de shape (bs,ni,nj,1). 
%\end{enumerate}
%} 

We'll now describe the composition of this scheme in a little more detail :
\begin{enumerate}[label=\textbullet]
	\item We start with input X of shape (batch\_size, height, width, nb\_channels) with batch\_size the number of images to be processed at the same time, height and width the dimensions of the images and nb\_channels the number of channels. Simplify by (bs,ni,nj,nk).
	\item We perform a $P$ transformation in order to move to a space with more channels. This step enables the network to build a sufficiently rich representation of the data.  For example, a Dense layer (also known as fully-connected) can be used. 	
	\item We then apply $L$ Fourier layers, noted $\mathcal{H}_\theta^l,\; l=1,\dots,L$, whose specifications will be detailed in Section \ref{FNO.fourierlayer}.
	\item We then return to the target dimension by performing a $Q$ transformation. In our case, the number of output channels is 1.
	\item We then obtain the output of the $Y$ model of shape (bs,ni,nj,1). 
\end{enumerate}

\modif{rajouter la valeur des paramètres dans un tableau : width,modes... }

\subsubsection{Fourier Layer structure} \label{FNO.fourierlayer}

%\trad{Chaque couche de Fourier est composé de deux sous-couches :}

Each Fourier layer is divided into two sublayers:

\begin{equation*}
	\tilde{Y}=\mathcal{H}_\theta^l(\tilde{X})=\sigma\left(\mathcal{C}_\theta^l(\tilde{X})+\mathcal{B}_\theta^l(\tilde{X})\right)
\end{equation*}

%\trad{où 
%\begin{enumerate}[label=\textbullet]
%	\item $\tilde{X}$ correspond à l'entrée de la couche courante et $\tilde{Y}$ à la sortie.
%	\item $\sigma$ est une fonction d'activation. Pour $l=1,\dots,L-1$, on prendra la fonction d'activation ReLU (Rectified linear unit) et pour $l=L$ on prendra la fonction d'activation GELU (Gaussian Error Linear Units).
%	\item $\mathcal{C}_\theta^l$ est une couche de convolution où la convolution est faite par FFT
%	\item $\mathcal{B}_\theta^l$ is the "bias-layer".
%\end{enumerate}}

where
\begin{enumerate}[label=\textbullet]
	\item $\tilde{X}$ corresponds to the input of the current layer and $\tilde{Y}$ to the output.
	\item $\sigma$ is an activation function. For $l=1,\dots,L-1$, we'll take the activation function ReLU (Rectified Linear Unit) and for $l=L$ we'll take the activation function GELU (Gaussian Error Linear Units).\modif{rajouter schéma + argument fcts d'activation}.
	\item $\mathcal{C}_\theta^l$ is a convolution layer where convolution is performed by FFT (Fast Fourier Transform).
	\item $\mathcal{B}_\theta^l$ is the "bias-layer".
\end{enumerate}

\paragraph{Convolution sublayer :}

%\trad{Chaque couche de convolution $\mathcal{C}_\theta^l$ contient un kernel $\hat{W}$ entraînable et effectue la transformation}

Each $\mathcal{C}_\theta^l$ convolution layer contains a trainable kernel $\hat{W}$ and performs the transformation

\begin{equation*}
	\mathcal{C}_\theta^l(X)=\mathcal{F}^{-1}(\mathcal{F}(X)\cdot\hat{W})
\end{equation*}

%\trad{où $\mathcal{F}$ correspond à la transformée de Fourier discrète (DFT) en 2D sur une grille de résolution $ni\times nj$}

where $\mathcal{F}$ corresponds to the 2D Discrete Fourier Transform (DFT) on a $ni\times nj$ resolution grid and
\begin{equation*}
	(Y\cdot\hat{W})_{ijk}=\sum_{k'}Y_{ijk'}\hat{W}_{ijk'}
\end{equation*}
In other words, this transormation is applied channel by channel.

\newpage

The 2D DFT is defined by :

%\mathcal{F}(X)_{xyk}=\frac{1}{ni}\frac{1}{nj}\sum_{x'=0}^{ni-1}\sum_{y'=0}^{nj-1}X_{x'y'k}e^{2i\pi\frac{xx'}{ni}\frac{yy'}{nj}}

\begin{equation*}
\mathcal{F}(X)_{ijk}=\frac{1}{ni}\frac{1}{nj}\sum_{i'=0}^{ni-1}\sum_{j'=0}^{nj-1}X_{i'j'k}e^{-2\sqrt{-1}\pi\left(\frac{ii'}{ni}+\frac{jj'}{nj}\right)}
\end{equation*}

The inverse of the 2D DFT is defined by :

\begin{equation*}
\mathcal{F}^{-1}(X)_{ijk}=\sum_{i'=0}^{ni-1}\sum_{j'=0}^{nj-1}X_{i'j'k}e^{2\sqrt{-1}\pi\left(\frac{ii'}{ni}+\frac{jj'}{nj}\right)}
\end{equation*}

We can easily show that $\mathcal{F}$ is the reciprocal function of $\mathcal{F}^{-1}$. We have 

\begin{align*}
	\mathcal{F}^{-1}(\mathcal{F}(X))_{ijk}&=\sum_{i'=0}^{ni-1}\sum_{j'=0}^{nj-1}\mathcal{F}(X)_{i'j'k}e^{2\sqrt{-1}\pi\left(\frac{ii'}{ni}+\frac{jj'}{nj}\right)} \\	
	&=\frac{1}{ni}\frac{1}{nj}\sum_{i'j'}\sum_{i''j''}X_{i''j''k}e^{-2\sqrt{-1}\pi\left(\frac{i'i''}{ni}+\frac{j'j''}{nj}\right)}e^{2\sqrt{-1}\pi\left(\frac{ii'}{ni}+\frac{jj'}{nj}\right)} \\
	&=\frac{1}{ni}\frac{1}{nj}\sum_{i''j''}X_{i''j''k}\color{blue}\sum_{i'j'}e^{2\sqrt{-1}\pi\frac{i'}{ni}(i-i'')}e^{2\sqrt{-1}\pi\frac{j'}{nj}(j-j'')}\color{black}\\
	& \qquad \qquad \qquad \qquad \qquad \qquad\color{blue}:= S \color{black}
\end{align*}

Thus
\begin{enumerate}[label=\textbullet]
	\item If $(i,j)=(i'',j'')$ : $S=\sum_{i',j'}1=ni\times nj$
	\item If $(i,j)\ne(i'',j'')$ : 
	\begin{align*}
		S&=\sum_{i'}\left(e^{\frac{2\sqrt{-1}\pi}{ni}(i-i'')}\right)^{i'}\sum_{j'}\left(e^{\frac{2\sqrt{-1}\pi}{nj}(j-j'')}\right)^{j'} \\
		&=\frac{1-\left(e^{\frac{2\sqrt{-1}\pi}{ni}(i-i'')}\right)^{ni}}{1-e^{\frac{2\sqrt{-1}\pi}{ni}(i-i'')}}\times \frac{1-\left(e^{\frac{2\sqrt{-1}\pi}{nj}(j-j'')}\right)^{nj}}{1-e^{\frac{2\sqrt{-1}\pi}{nj}(j-j'')}} \\
		&=\frac{1-e^{2\sqrt{-1}\pi(i-i'')}}{1-e^{\frac{2\sqrt{-1}\pi}{ni}(i-i'')}}\times \frac{1-e^{2\sqrt{-1}\pi(j-j'')}}{1-e^{\frac{2\sqrt{-1}\pi}{ni}(j-j'')}}=0
	\end{align*}
	because if $i\ne i''$
	$$e^{2\sqrt{-1}\pi(i-i'')}=cos(2\pi(i-i''))+\sqrt{-1}sin(2\pi(i-i'')) = 1+0 =1$$
	and if $j\ne j''$
	$$e^{2\sqrt{-1}\pi(j-j'')}=cos(2\pi(j-j''))+\sqrt{-1}sin(2\pi(j-j'')) = 1+0 =1$$
\end{enumerate}

We deduce that
\begin{equation*}
	\mathcal{F}^{-1}(\mathcal{F}(X))_{ijk} = \frac{1}{ni}\frac{1}{nj} \times ni\times nj\times X_{ijk} = X_{ijk}
\end{equation*}

And finally $\mathcal{F}$ is the reciprocal function of $\mathcal{F}^{-1}$.

For more details about the Convolution sublayer, see Section \ref{FNO.details_conv}.

\paragraph{Bias subLayer :}

%\trad{La bias layer est une convolution 2D avec un kernel-size de 1. C'est-à-dire que cela ne fait qu'une multiplication matricielle sur les chanels, mais pixel par pixel. Autrement dit, elle mélange les channels par un kernel mais ne permet pas d'interaction entres les pixels.}

The bias layer is a 2D convolution with a kernel size of 1. This means that it only performs matrix multiplication on the channels, but pixel by pixel. In other words, it mixes channels via a kernel, but does not allow interaction between pixels.

Precisly,

\begin{equation*}
	\mathcal{B}_\theta^l(X)_{ijk}=\sum_{k'}X_{ijk}W_{k'k}+B_k
\end{equation*}

\subsubsection{Some details on the convolution sublayer} \label{FNO.details_conv}

In this section, we will specify some details for the convolution layer.

\paragraph{FFT :}

To speed up computations, we will use the FFT (Fast Fourier Transform). The FFT is a fast algorithm to compute the DFT. It is recursive : The transformation of a signal of size $N$ is make from the decomposition of two sub-signals of size $N/2$. The complexity of the FFT is $N\log(N)$ whereas the natural algorithm, which is a matrix multiplication, has a complexity of $N^2$.

\paragraph{Border issues :}

Let $W=\mathcal{F}^{-1}(\hat{W})$, we have :
\begin{equation*}
	\mathcal{C}_\theta^l(\tilde{X})=\mathcal{F}^{-1}\left(\mathcal{F}(X)\cdot\hat{W}\right)=\tilde{X}\star W
\end{equation*}
with
\begin{equation*}
	(\tilde{X}\star W)_{ij}=\sum_{i'j'}\tilde{X}_{i-i'[ni],j-j'[nj]}W_{i'j'}
\end{equation*}

In other words, multiplying in Fourier space is equivalent to performing a $\star$ circular convolution in real space. But these modulo operations are only natural for periodic images, which is not our case. \modif{Expliquer padding !}

\paragraph{Real DFT :}

In reality, we'll be using a specific implementation of FFT, called RFFT (Real Fast Fourier Transorm).In fact, for $\mathcal{F}^{-1}(A)$ to be real if $A$ is a complex-valued matrix, it is necessary that A respects the Hermitian symmetry:
\begin{equation*}
	A_{i,nj-j} = \bar{A}_{i,j}
\end{equation*}
In our case, we want $\mathcal{C}_\theta^l(X)$ to be a real image, so $\mathcal{F}(X)\cdot\hat{W}$ must verify Hermitian-symmetry.

%\trad{Pour celà, il nous suffit de récupérer seulement la moitié des coefficients de Fourier Discret (DFC) $\mathcal{F}(X)_{ijk}$ et l'autre moitié sera déduit par la symétrie hermitienne. Plus précisément, en utilisant l'implémentation spécifique RFFT, les DFC sont stockés dans une matrice de taille $(ni,nj//2+1)$. On peut alors effectuer la multiplication par le kernel $\hat{W}$ puis  lorsque l'on effectue le RFFT inverse, les DFC seront automatiquement symétrisé. Ainsi la symétrie hermitienne de $\mathcal{F}\cdot\hat{W}$ est vérifiée et $\mathcal{C}_\theta^l(X)$ est bien une image réelle.}

To do this, we only need to collect half of the Discrete Fourier Coefficients (DFC) and the other half will be deduced by Hermitian symmetry. More precisely, using the specific RFFT implementation, the DFCs are stored in a matrix of size $(ni,nj//2+1)$. Multiplication can then be performed by the $\hat{W}$ kernel, and when the inverse RFFT is performed, the DFCs will be automatically symmetrized. So the Hermitian symmetry of $\mathcal{F}(X)\cdot\hat{W}$ is verified and $\mathcal{C}_\theta^l(X)$ is indeed a real image. \modif{ajouter schéma exemple ?}

\paragraph{Low pass filter :}

%\trad{Quand on effectue une DFT sur une image, les DFC associés aux hautes fréquences sont en pratique très faibles. C'est pourquoi, on peut facilement filtrer une image en ignorant ces hautes fréquences, c'est-à-dire en tronquant les hauts modes de Fourier. En fait, l'élimination des modes de Fourier supérieurs permet une sorte de régularisation qui aide à la généralisation. Ainsi, en pratique, il est suffisant de ne garder que les DFC correspondant aux basses fréquences. Typiquement, pour des images de résolution $32\times 32$ à $128\times 128$, on peut conserver seulement les $20\times 20$ DFC associés aux basses fréquence. }

When we perform a DFT on an image, the DFCs related to high frequencies are in practice very low. This is why we can easily filter an image by ignoring these high frequencies, i.e. by truncating the high Fourier modes. In fact, eliminating the higher Fourier modes enables a kind of regularization that helps the generalization. So, in practice, it's sufficient to keep only the DFCs corresponding to low frequencies. Typically, for images of resolution $32\times 32$ to $128\times 128$, we can keep only the $20\times 20$ DFCs associated to low frequencies.

\modif{ajouter schéma ?}

\subsection{Application}

\trad{Dans notre cas, on souhaite apprendre au FNO à prédire des solutions d'EDP. Plus précisément, on souhaite que le réseau soit capable de prédire la solution à partir d'un terme source $f$. 
\modif{enrichissement des données, grilles régulières}}
	
	\newpage
	\section{Correction}

\textbf{PLAN Correction :} expliquer le contexte général (on souhaite corriger la sortie d'un FNO et faire référence à 3.4 Application où il faut mettre un schéma) puis expliquer en intro le cheminement (résultats sr sol analytique -> sol ana + pert -> FNO -> pb FNO -> test Legendre (car P10) -> rajouter trop d'erreurs donc Multiperceptron -> pb dérivées donc PINNS (on pourrait inclure les dérivées dans la loss)) + expliquer que le but est de considérer PhiFEM mais que certains résultats théoriques et pratiques sont avec FEM standard \modif{rajouter si c'est fait résultat correction avec $\tilde{\phi}$ sol phifem dans intro.}

\modif{rajouter rq quelque part avec librairies utilisées : FEniCS, Tensorflow, Pytorch !}

\modif{Pourquoi entrainement maillage fin puis correction maillage grossier ?}

\begin{enumerate}
	\item \textbf{Présentation des différentes méthodes de Correction considérées :} présentation des différentes méthodes de correction (+ résultats analytiques en annexe ? comme "papier" rehaussement etc)  + ajouter schéma ?
	
	\modif{rq "précise" sur en quoi ça aide le solveur de lui fournir une solution proche de la solution exacte (fcts de base ?)}
	
	\item \textbf{Présentation des problèmes considérés :} Comme dit précédemment, on ne considérera ici que le pb de Poisson... + expliquer domaines (cercle et carré) + expliquer sol considérées (u trigo, f gaussienne => sol sur-raffinée, u polynomial pour PINNs) \modif{A faire en dernier en ne mettant que les trucs qu'on a considéré ensuite !!}
	
	\item \textbf{Différents résultats de correction}
	
	\begin{enumerate}[label=\textbullet]
		\item \textbf{Correction sur solution exacte :} Résultat sur sol exacte avec FEM et PhiFEM
		
		\item \textbf{Correction sur solution perturbée :} 
		\begin{itemize}
			\item solution perturbée manuellement
			\item correction sur PhiFEM : Pas encore tester mais intéressant \modif{à tester !!!}
		\end{itemize}
		
		\item \textbf{Correction avec FNO :} loss sur w (ou phi w)... + résultats Phifem P2 puis Legendre P10 (ajoute trop d'erreurs ?)
		
		\item \textbf{Correction avec d'autres réseaux :} expliquer pourquoi on fait ça (comprendre si la correction fonctionne sur des réseaux plus simple et donc ça pourrait confirmer qu'il y a bien un problème au niveau de la perturbation créé par le FNO) + rajouter première implémentation tensorflow puis utilisation d'un travail en cours (avec pytorch)
		\begin{itemize}
			\item \textbf{Multiperceptron :} présentation rapide avec schéma (+références) + résultats obtenus + expliquer qu'on doit rajouter l'apprentissage des dérivées car elles sont mauvaises et utilisées dans le solveur (dans la formulation variationnelle).
			\item \textbf{PINNs :} présentation rapide avec schéma ? + résultats obtenus pour $f=1$ (+rajouter rq que par manque de temps, on n'a pas trouvé la bonne architecture du PINNs pour apprendre la solution trigo et donc on n'a considéré pour l'instant que la correction pour $f=1$)
		\end{itemize} 
	\end{enumerate}

\end{enumerate}

\modif{+ rajouter rq ou dans conclusion pour expliquer qu'on pourrait également tester de faire varier le type de solveur matricielle dans les solveurs pour la correction !}

\subsection{Presentation of the different problem considered}

\modif{A faire à la fin !}

\subsection{Presentation of the different correction methods considered} \label{Corr.methods}


\modif{ajouter formulation variationnelle : FEM + PhiFEM ?}

Here we are given $\tilde{\phi}$ an "initial" solution to the problem under consideration, i.e. a solution that has not yet been corrected. This may be a perturbed analytic solution, a $\phi$-FEM solution, or a solution predicted by a neural network (such as an FNO, a Multi-perceptron network or PINNs, for example). The aim is to reinject this solution into a new problem in order to improve the accuracy of the solution. To achieve this, we consider 3 types of correction: correction by addition (Section \ref{Corr.method.add}), correction by multiplication (Section \ref{Corr.method.mult}) and correction by multiplication on an elevated problem (Section \ref{Corr.method.mult_reh}).

\subsubsection{Correction by adding} \label{Corr.method.add}

In this first method, we will try to approximate the solution obtained $\tilde{\phi}$ to the exact solution by completing the difference between the two, which is what we will call correction by adding. To do this, we will consider
\begin{equation*}
	\tilde{u}=\tilde{\phi}+C
\end{equation*}
and we want to find $C: \Omega \rightarrow \mathbb{R}^d$ solution to the problem
\begin{equation*}
	\left\{\begin{aligned}
		-\Delta \tilde{u}&=f, \; &&\text{on } \Omega, \\
		\tilde{u}&=g, \; &&\text{in } \Gamma.
	\end{aligned}\right.
\end{equation*}
\begin{Rem}
	Note that this problem is in fact equivalent to the initial \modif{add ref} problem. We only hope that the approximate solution $tilde{u}$ obtained is more accurate than the approximate solution $u$ obtained by solving the initial problem.
\end{Rem}
Rewriting the problem, we seek to find $C: \Omega \rightarrow \mathbb{R}^d$ solution to the problem
\begin{equation}
\label{eq.corr.pbc_add}
\left\{\begin{aligned}
	-\Delta C&=\tilde{f}, \; &&\text{on } \Omega, \\
	C&=0, \; &&\text{in } \Gamma.
\end{aligned}\right. \tag{$\mathcal{C}_{+}$}
\end{equation}
with $\tilde{f}=f+\Delta\tilde{\phi}$.

\subsubsection{Correction by multiplying} \label{Corr.method.mult}

In this second method, we try to approach the exact solution in a different way. In fact, we want to bring the factor between the $\tilde{\phi}$ solution and the solution of the corrected problem closer to 1. In other words, by considering 
\begin{equation*}
	\tilde{u}=\tilde{\phi}C,
\end{equation*}
we try to bring $C=\frac{\tilde{u}}{\tilde{\phi}}$ closer to 1 (for $\tilde{\phi}\ne 0$). This type of correction is called correction by multiplying.

So we're looking for $C: \Omega \rightarrow \mathbb{R}^d$ solution to the problem
\begin{equation}
	\label{eq.corr.pbc_mult}
	\left\{\begin{aligned}
		&-\Delta (\tilde{\phi}C)=f \quad &&\Omega \\
		&C=1 \quad &&\Gamma
	\end{aligned}\right. \tag{$\mathcal{C}_\times$}
\end{equation}

\begin{Rem}
	In the same way as for correction by adding, we note that this problem is equivalent to the initial \modif{add ref} problem.
\end{Rem}

\subsubsection{Correction by multiplying on an elevated problem} \label{Corr.method.mult_reh}

We now introduce a third correction method, which we'll call multiplication correction on an elevated problem. This method is in fact very similar to the previous one (correction by multiplication), except that we are no longer trying to correct the same problem.

The initial modified problem, which we now consider, consists in finding $u : \Omega \rightarrow \mathbb{R}^d$ such that
\begin{equation}
	\label{eq.corr.pb_reh}
	\left\{
	\begin{aligned}
		-\Delta \hat{u} = f, \; &&\text{in } \; \Omega, \\
		\hat{u}=g+m, \; &&\text{on } \; \Gamma,
	\end{aligned}
	\right. \tag{$\mathcal{P}^\mathcal{M}$}
\end{equation}
with $\hat{u}=u+m$ and $m$ a constant.

\night{ajouter schéma (à gauche une solution initiale simple et à droite la solution rehaussée.)}

We then apply the same multiplication correction method, but this time on the modified problem, which has been elevated by a constant $m$. We then consider
\begin{equation*}
	\tilde{u}=\hat{\phi}C
\end{equation*}
avec 
\begin{equation*}
	\hat{\phi}=\tilde{\phi}+m
\end{equation*}
and so we look for $C: \Omega \rightarrow \mathbb{R}^d$ solution to the problem
\begin{equation}
	\label{eq.corr.pbc_mult_reh}
	\left\{\begin{aligned}
		&-\Delta (\hat{\phi}C)=f \quad &&\Omega \\
		&C=1 \quad &&\Gamma
	\end{aligned}\right. \tag{$\mathcal{C}_\times^\mathcal{M}$}
\end{equation}

\begin{Rem}
	Note that if the problem is sufficiently elevated for its solution to be strictly positive, the operation of bringing $C=\frac{\tilde{u}}{\hat{\phi}}$ closer to 1 doesn't pose a problem (since in this case $\hat{\phi}\ne 0$). Moreover, we can easily return to the original problem by subtracting $m$ from $\tilde{u}$. In this way, by correcting the elevated problem by multiplication, we can easily return to correcting the initial problem by multiplication. \modif{En annexe, on a un document expliquant l'intérêt de rehausser le problème +add ref.} \modif{rajouter également document qui explique qu'au final on s'est rendu compte que correction par multiplication sur problème rehaussé $\iff$ correction par addition + ajouter document}
\end{Rem}

\subsection{Different correction results}

As explained above, we wish to combine $\phi$-FEM and FNO in order to predict the solution of the Poisson problem as accurately as possible. In this section, we present various results obtained using the 3 correction methods presented in the previous section (Section \ref{Corr.methods}). It is important to note that, for practical purposes, almost all the following results obtained with $\phi$-FEM will be compared with those obtained with the standard FEM method.

We'll start by presenting the results obtained on an analytical solution (Section \ref{Corr.results.ana}). We'll consider here the "initial" solution $\tilde{\phi}$, which we'll inject into the correction problems, as the analytical solution of the problem. This first step simply enables us to check that, by supplying the exact solution directly to the correction solvers, they are indeed reduced to machine errors.

Next, in order to verify that correction solvers can improve accuracy when providing a solution close to the exact solution, we will consider the case of so-called "disturbed" solutions (Section \ref{Corr.results.disturbed}). This step will also provide us with a basis for further work, giving us an idea of what we can expect in terms of neural network output correction.

Finally, we'll consider the case of neural networks with an FNO in Section \ref{Corr.results.FNO} and then with a multi-perceptron network and PINNs in Section \ref{Corr.results.neural_net}. The reasons for considering other neural networks will be explained in detail in these sections.

\subsubsection{Correction on exact solution} \label{Corr.results.ana}

\subsubsection{Correction on disturbed solution} \label{Corr.results.disturbed}

\paragraph{Manually perturbed solution}
\paragraph{correction on PhiFEM}

\subsubsection{Correction with FNO} \label{Corr.results.FNO}

\subsubsection{Correction with other networks} \label{Corr.results.neural_net}
\paragraph{Multiperceptron}
\paragraph{PINNs}
	
	\newpage
	\section{Conclusion}

\modif{Penser à parler de la thèse : sujet etc...}

\modif{TO COMPLETE !}
	
%	\newpage
%	\section*{Bibliography}
%	\addcontentsline{toc}{section}{Bibliography}
%		
%	\printbibliography
	
\end{document}