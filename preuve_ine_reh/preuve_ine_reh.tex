\documentclass[french]{article}
\usepackage[T1]{fontenc}
\usepackage[utf8]{inputenc}
\usepackage[french]{babel}
\usepackage{amsmath}
\usepackage{mathtools}
\usepackage{color}
\usepackage[svgnames,dvipsnames]{xcolor} 
\usepackage{soul}
\usepackage{amssymb}
\usepackage{enumitem}
\usepackage{multicol}
\usepackage[left=2cm,right=2cm,top=2cm,bottom=2cm]{geometry}
\newcommand{\mathcolorbox}[2]{\colorbox{#1}{$\displaystyle #2$}}
\usepackage{pifont}
\usepackage{pst-all}
\usepackage{pstricks}
\usepackage{delarray}
\usepackage{setspace}
\usepackage{graphicx}
\usepackage{hyperref}
\usepackage{nicematrix}
\usepackage{listings}
\usepackage{float}

\hypersetup{
	colorlinks=true,
	linkcolor=blue,
	filecolor=magenta,      
	urlcolor=cyan,
	pdfpagemode=FullScreen,
}

\usepackage{amsthm}
\newtheorem*{Rem}{Remarque}

\newenvironment{conclusion}[1]{%
	\begin{center}\normalfont\textbf{Conclusion}\end{center}
	\begin{quotation} #1 \end{quotation}
}{%
	\vspace{1cm}
}

\lstset{language=C++,
	basicstyle=\ttfamily,
	keywordstyle=\color{blue}\ttfamily,
	stringstyle=\color{red}\ttfamily,
	commentstyle=\color{green}\ttfamily,
	morecomment=[l][\color{magenta}]{\#}
}

\setlength\parindent{0pt}

\begin{document}
	LECOURTIER Frédérique \hfill \today
	\begin{center}
		\Large\textbf{{Estimation d'erreur - Problème rehaussé.}}\\
	\end{center}

	On se place ici dans le cadre de FEM standard. 
	
	\textbf{Problème initital :} \\
	On considère initialement le problème de Poisson avec condition de Dirichlet homogène ou non homogène :
	\begin{equation}
		\left\{\begin{aligned}
			&-\Delta u=f \quad &&\text{dans } \Omega \\
			&u=g \quad &&\text{sur } \Gamma
		\end{aligned}\right. \label{pb_initial} \tag{$\mathcal{P}$}
	\end{equation}

	\textbf{Problème considéré :} \\
	Dans notre cas, on souhaite appliquer une correction à la sortie d'un FNO.
	On va considérer ici que l'on possède une solution analytique $u_{ex}$ et qu'après une utilisation du FNO, on obtient une solution du type
	$$\tilde{\phi}(x,y) = u_{ex}(x,y)-\epsilon P(x,y)$$
	avec $P$ la perturbation (tel que $P=0$ sur $\Gamma$) et $\epsilon$ petit.
	
	On considère
	$$\hat{\phi}=\tilde{\phi}+m=u_{ex}-\epsilon P+m=\widehat{u_{ex}}-\epsilon P$$
	avec $\widehat{u_{ex}}=u_{ex}+m$ et $m$ une constante.
	
	On souhaite alors résoudre le problème suivant :
	\begin{equation}
		\left\{\begin{aligned}
			&-\Delta (\hat{\phi}C)=f \quad &&\Omega \\
			&(\hat{\phi}C)=g+m \quad &&\Gamma
		\end{aligned}\right. \label{pb_reh} \tag{$\mathcal{C}$}
	\end{equation}

	On pose alors
	$$\hat{u}=\hat{\phi}C$$

	\textbf{But du document :} \\
	Démonter la propriété suivante :
	\begin{equation}
		\left|\left|\hat{u}-\hat{u_h}\right|\right|_0\le ch^{k+1}||\hat{\phi}||_\infty\left|C\right|_{k+1}
		\label{ine_a_dem}
	\end{equation}
	
	\textbf{Problèmes variationnels :} \\
	
	Problème variationnel :
	$$\text{Trouver } \hat{u}\in V \text{ tel que } a(\hat{u},v)=l(v), \;\forall v\in V$$
	
	Problème variationnel approché :
	$$\text{Trouver } \hat{u_h}\in V_h \text{ tel que } a(\hat{u_h},v_h)=l(v_h), \;\forall v_h\in V_h$$
	
	\section{Partie 1 : norme $H^1$}
	
	Comme $V_h$ est un sous espace vectoriel de $V$, en posant $v=v_h$, on obtient :
	$$a(\hat{\phi}C,\hat{\phi}v_h)-a(\hat{\phi}C_h,\hat{\phi}v_h)=0 \quad \forall v_h\in V_h$$
	On a alors l'orthogonalité de Galerkin : \color{red}{(ATTENTION : Abus de notation sur $v_h$ !)}\color{black}
	$$a(\hat{u}-\hat{u_h},v_h)=0 \quad \forall v_h\in V_h$$
	On a alors
	\begin{align*}
		\color{red}{\nu}\color{black}||\hat{u}-\hat{u_h}||_1^2&\le\alpha a(\hat{u}-\hat{u_h},\hat{u}-\hat{u_h}) &&\text{par coercivité} \\
		&=\alpha a(\hat{u}-\hat{u_h},\hat{u}-I_h\hat{u}+I_h\hat{u}-\hat{u_h}) \\
		&=\alpha a(\hat{u}-\hat{u_h},\hat{u}-I_h\hat{u}) &&\text{par orthogonalité de Galerkin en prenant } v_h=\hat{u_h}-I_h\hat{u} \\
		&\le\alpha |\hat{u}-\hat{u_h}|_1|\hat{u}-I_h\hat{u}|_1 &&\text{par continuité} \\
		&\le\alpha ||\hat{u}-\hat{u_h}||_1|\hat{u}-I_h\hat{u}|_1
	\end{align*}
	Ainsi
	$$||\hat{u}-\hat{u_h}||_1\le\alpha|\hat{u}-I_h\hat{u}|_1$$
	Or 
	$$|\hat{u}-I_h\hat{u}|_1=|(C-I_hC)\hat{\phi}|_1$$
	En posant $A=C-I_hC$, on a
	$$|A\hat{\phi}|_1=||(A\hat{\phi})'||_0=||A'\hat{\phi}+A\hat{\phi}'||_0\le||A'\hat{\phi}||_0+||A\hat{\phi}'||_0$$
	Comme
	$$||A'\hat{\phi}||_0=\sqrt{\int_\Omega(A'\hat{\phi})^2}\le \max_\Omega \hat{\phi}\sqrt{\int_\Omega(A')^2}=||\hat{\phi}||_\infty|A|_1\le||\hat{\phi}||_\infty||A||_1$$
	et
	$$||A\hat{\phi}'||_0=\sqrt{\int_\Omega(A\hat{\phi}')^2}\le \max_\Omega \hat{\phi}'\sqrt{\int_\Omega(A)^2}=||\hat{\phi}'||_\infty||A||_0\le \alpha||\hat{\phi}||_\infty||A||_1$$
	Ainsi
	$$|A\hat{\phi}|_1\le\alpha ||\hat{\phi}||_\infty||A||_1$$
	Donc
	$$|\hat{u}-I_h\hat{u}|_1=|(C-I_hC)\hat{\phi}|_1\le\alpha ||\hat{\phi}||_\infty||C-I_hC||_1$$

	Finalement en utilisant l'inégalité d'interpolation, on obtient
	\begin{equation}
		\boxed{||\hat{u}-\hat{u_h}||_1\le\alpha h^k ||\hat{\phi}||_\infty |C|_{k+1}}
		\label{norme_H1}
	\end{equation} 

	\section{Partie 2 : norme $L^2$}
	
	On applique la méthode de dualité d'Aubin-Nitsche. On considère le problème dual : \\
	Soit $\hat{z}\in H_0^1(\Omega)$ solution du problème
	$$\left\{\begin{aligned}
		&-\Delta\hat{z}=\hat{e_h} \quad &&\text{dans }\Omega \\
		&\hat{z}=0 \quad &&\text{sur } \Gamma
	\end{aligned}\right.$$
	avec $\hat{e_h}=\hat{u}-\hat{u_h}$.
	Alors
	$$a(u,v)=-\int_\Omega\Delta u\cdot v=\int_\Omega\nabla u\cdot\nabla v$$
	et comme $e_h\in H_0^1(\Omega)$
	\begin{equation}
		a(\hat{z},\hat{e_h})=\int_\Omega(-\Delta \hat{z})\cdot \hat{e_h}=\int_\Omega \hat{e_h}^2=||\hat{e_h}||_0^2
		\label{norme_e_h}
	\end{equation}
	De plus, par les propriétés de régularité : 
	\begin{equation}
		\hat{z}\in H^2(\Omega)
		\label{z_H_2}
	\end{equation}
	et
	\begin{equation}
		||\hat{z}||_2\le \alpha||\hat{e_h}||_0 =\alpha||\hat{u}-\hat{u_h}||_0
		\label{prop_reg}
	\end{equation}
	Ainsi
	\begin{align*}
		||\hat{u}-\hat{u_h}||_0^2&=||\hat{e_h}||_0^2 \\
		&=a(\hat{z},\hat{e_h}) &&\text{par \ref{norme_e_h}} \\
		&=a(\hat{z}-I_h\hat{z},\hat{e_h}) &&\text{par orthogonalité de Galerkin} \\
		&\le\alpha|\hat{z}-I_h\hat{z}|_1|\hat{e_h}|_1 &&\text{par continuité} \\
		&\le\alpha h|\hat{z}|_2|\hat{e_h}|_1 &&\text{par \ref{z_H_2} et par inégalité d'interpolation} \\
		&\le\alpha \cdot h|\hat{z}|_2 \cdot h^k||\hat{\phi}||_\infty |C|_{k+1} &&\text{par \ref{norme_H1}} \\
		&\le\alpha h^{k+1}||\hat{u}-\hat{u_h}||_0||\hat{\phi}||_\infty |C|_{k+1} &&\text{par \ref{prop_reg}}
	\end{align*}
	Finalement
	\begin{equation}
		\boxed{||\hat{u}-\hat{u_h}||_0\le\alpha h^{k+1}||\hat{\phi}||_\infty |C|_{k+1}}
	\end{equation}

	\begin{Rem}
		On notera que
		$$||\hat{u}-\hat{u_h}||_0=||u+m-(u_h+m)||_0=||u-u_h||_0$$
	\end{Rem}
\end{document}