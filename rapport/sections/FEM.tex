\section{Finite Element Methods (FEMs)}

In the following, we will consider the Poisson problem with Dirichlet condition (homogeneous or inhomogeneous):

\textbf{Problem :} Find $u : \Omega \rightarrow \mathbb{R}^d$ such that

\begin{equation*}
	\left\{
		\begin{aligned}
			-\Delta u &= f \; &&\text{in } \; \Omega \\
			u&=g \; &&\text{on } \; \partial\Omega
		\end{aligned}
	\right.
\end{equation*}

with $\Delta$ the Laplace operator and $\Omega\subset\mathbb{R}^d$ a lipschitzian bounded open set (and $\partial\Omega$ its boundary).

\textbf{Associated physical model :} Newtonian gravity, Electrostatics, Fluid dynamics...

\subsection{Standard FEM}

\subsubsection{Some notions of functional analysis.}

\modif{AJOUTER : Déf Espace de Hilbert (+ ce qu'il faut pour def ça : Espace de Soboloev ? ... ) + Déf $L^2$,}

\subsubsection{General principle of the method}

Let's consider a domain $\Omega$ whose boundary is denoted $\partial\Omega$. We seek to determine a function $u$ defined on $\Omega$, solution of a partial differential equation (PDE) for given boundary conditions.

The general approach of the finite element method is to write down the variational formulation of this PDE, thus giving us a problem of the following type:

\textbf{Variational Problem :}
\begin{equation*}
	\text{Find } u\in V \text{ such that } a(u,v)=l(v), \;\forall v\in V
\end{equation*}

where $V$ is a Hilbert space, $a$ is a bilinear form and $l$ is a linear form.

To do this, we multiply the PDE by a test function $v\in V$, then integrate over $L^2(\Omega)$.

The idea of FEM is to use Galerkin's method. We then look for an approximate solution $u_h$ in $V_h$, a finite-dimensional space dependent on a positive parameter $h$ such that

\begin{equation*}
	V_h\subset V, \quad \dim V_h = N_h<\infty, \quad \forall h>0
\end{equation*}

The variational problem can then be approached by :

\textbf{Approach Problem :}
\begin{equation*}
	\text{Find } u_h\in V_h \text{ such that } a(u_h,v_h)=l(v_h), \;\forall v_h\in V
\end{equation*}

As $V_h$ is of finite dimension, we can consider a basis $(\varphi_1,\dots,\varphi_{N_h})$ of $V_h$ and thus decompose $u_h$ on this basis as :

\begin{equation}
	\label{decomp1}
	u_h=\sum_{i=1}^{N_h}u_i\varphi_i	
\end{equation}

The approached problem is then rewritten as

\begin{equation*}
	\text{Find } u_1,\dots,u_{N_h} \text{ such that } \sum_{i=1}^{N_h}u_i a(\varphi_i,v_h)=l(v_h), \;\forall v_h\in V 
\end{equation*}

and

\begin{equation*}
	\text{Find } u_1,\dots,u_{N_h} \text{ such that } \sum_{i=1}^{N_h}u_i a(\varphi_i,\varphi_j)=l(\varphi_j), \;\forall j\in \{1,\dots,N_h\}
\end{equation*}

Solving the PDE involves solving the following linear system:
\begin{equation*}
	AU=b
\end{equation*}
with
\begin{equation*}
	A=(a(\varphi_i,\varphi_j))_{1\le i,j\le N_h}, \quad U=(u_i)_{1\le i\le N_h} \quad \text{and} \quad b=(l(\varphi_j))_{1\le j\le N_h}
\end{equation*}

\subsubsection{Some details on FEM}

\modif{Notions à aborder : ef de Lagrange + unisolvance + maillage + Transformation géométrique}

After having seen the general principle of FEM, it remains to define the $V_h$ spaces and the $\{\varphi_i\}$ basis functions.

\begin{Rem}
	The choice of $V_h$ space is fundamental to having an efficient method that gives a good approximation $u_h$ of $u$. In particular, the choice of the $\{\varphi_i\}$ basis of $V_h$ influences the structure of the $A$ matrix in terms of its sparsity and condition number.
\end{Rem}

\modif{TO COMPLETE !}

%\paragraph{Finite Lagrange Element}
%
%\modif{A COMPLETER !}
%
%\paragraph{Mesh}
%
%\trad{La première étape consiste à construire un maillage de $\Omega$.}
%
%\begin{Rem}
%	\trad{La partie de génération de maillage est une étape importante }
%\end{Rem}

\subsubsection{Application to the Poisson problem}

\modif{Ajouter formulation variationnel Poisson}

\begin{Prop}[Lax-Milgram]
	
	Let $a$ be a continuous, coercive bilinear form on $V$ and $l$ a continuous, linear form on $V$. Then the variational problem has a unique solution $u\in V$. 
	
	Moreover, if the bilinear form is symmetrical, $u$ is a solution to the following minimization problem:
	\begin{equation*}
		J(u)=\min_{v\in V} J(v), \quad J(v)=\frac{1}{2}a(v,v)-l(v)
	\end{equation*}
\end{Prop}

It can then be shown that the Poisson problem with Dirichlet condition has a unique weak solution.

\modif{rajouter preuve}

\subsection{$\phi$-FEM}

%The Φ-FEM method has an advantage that is very interesting in the context of organ geometries. Indeed, this type of geometry can deform in time and meshing a fictitious domain around this geometry avoids having to remesh the geometry in time. Thus only the levelset function will be modified and the mesh can be fixed. Moreover, a Cartesian mesh of the fictitious domain allows us to use the same type of neural network as those applied to images: this is the subject that will be approached during the internship.

