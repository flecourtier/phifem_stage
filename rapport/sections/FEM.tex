\section{Finite Element Methods (FEMs)}

%\subsection{Problème considéré}

%Pb de Poisson

\trad{On considérera dans toute la suite, le problème de Poisson avec condition de Dirichlet homogène.}

\subsection{Principle of the standard finite element method}

\trad{On considère un domaine $\Omega$ dont la frontière est notée $\partial\Omega$. On cherche à déterminer une fonction $u$ définie sur $\Omega$, solution d'une équation aux dérivées partielles (EDP) pour des conditions aux limites données.}

\subsection{$\phi$-FEM}

%The Φ-FEM method has an advantage that is very interesting in the context of organ geometries. Indeed, this type of geometry can deform in time and meshing a fictitious domain around this geometry avoids having to remesh the geometry in time. Thus only the levelset function will be modified and the mesh can be fixed. Moreover, a Cartesian mesh of the fictitious domain allows us to use the same type of neural network as those applied to images: this is the subject that will be approached during the internship.

