\section{Semaine 13 : 08/05/2023 - 12/05/2023}
\graphicspath{{semaines/semaine_13/images/}}

\setcounter{equation}{0}
\setlength\parindent{0pt}

\subsection{Comparaison de 2 méthodes de correction \faBookmarkO}

\begin{abstract}
	On considère le problème de Poisson avec condition de Dirichlet homogène ou non homogène :
	\begin{equation}
		\label{pb1}
		\left\{\begin{aligned}
			&-\Delta u=f \quad &&\Omega \\
			&u=g \quad &&\Gamma
		\end{aligned}\right. \tag{$\mathcal{E}_1$}
	\end{equation}
	
	On a ainsi une EDP que l'on souhaite résoudre sur un domaine $\Omega$. On note $\Gamma$ le bord de $\Omega$, c'est-à-dire $\Gamma=\partial\Omega$. 
	
	Dans notre cas, on souhaite appliquer une correction à la sortie d'un FNO.
	On considère ici que l'on possède une solution analytique $u$ et qu'après une utilisation du FNO, on obtient une solution du type
	\begin{equation*}
		\label{phi_tild}
		\tilde{\phi}(x,y)=u_p(x,y) = u(x,y)-\epsilon P(x,y)
	\end{equation*}
	avec $P$ la perturbation (tel que $P=0$ sur $\Gamma$) et $\epsilon$ petit.
	
	Ce document a pour but de comparer deux méthodes de correction de la solution obtenue.
\end{abstract}

\subsubsection{Correction avec FEM}

\paragraph{Présentation des 2 méthodes}

\begin{enumerate}[label=\textbullet]
	\item \textbf{Méthode 1 : }
	On souhaite résoudre le problème suivant
	\begin{equation}
		\label{pbc1}
		\left\{\begin{aligned}
			&-\Delta (\tilde{\phi}C)=f \quad &&\Omega \\
			&C=1 \quad &&\Gamma
		\end{aligned}\right. \tag{$\mathcal{C}_1$}
	\end{equation}
	
	avec $\tilde{u}=\tilde{\phi}C$.
	
	Dans un autre document, on a présenté l'intérêt de rehausser le problème et de se ramener au problème suivant
	
	\begin{equation}
		\left\{\begin{aligned}
			&-\Delta (\hat{\phi}C)=f \quad &&\Omega \\
			&\hat{u}=g+m \quad &&\Gamma
		\end{aligned}\right. \label{pbc1r} \tag{$\mathcal{C}_1^\mathcal{R}$}
	\end{equation}
	avec $\hat{u}=\hat{\phi}C+m$ où $\hat{\phi}=\tilde{\phi}+m$ ($m$ une constante).
	
	
	\item \textbf{Méthode 2 : } On souhaite résoudre le problème suivant
	\begin{equation}
		\label{pbc2}
		\left\{\begin{aligned}
			&-\Delta C=\tilde{f} \quad &&\Omega \\
			&C=0 \quad &&\Gamma
		\end{aligned}\right. \tag{$\mathcal{C}_2$}
	\end{equation}
	
	avec $\tilde{u}=\tilde{\phi}+C$ et $\tilde{f}=f+\Delta\tilde{\phi}$.
	
	\begin{Rem}
		On notera que dans ce cas rehausser le problème n'a aucun intérêt. \\
		En effet, la décomposition de $C_h$ sur la base $(\varphi_1,\dots,\varphi_{N_h})$ de $V_h$ s'écrit pour ce problème			
		$$C_h=\sum_{i=1}^{N_h}C_i\varphi_i$$
		avec $C_i=u(x_i)-\tilde{\phi}(x_i)$.
		Et donc, on a l'inégalité suivante
		$$||(u-\tilde{\phi})-C_h||_{L^2(\Omega)}\le ch^{k+1}|u-\tilde{\phi}|_{H^{k+1}(\Omega)}$$
		Alors
		$$|u-\tilde{\phi}|_{H^{k+1}(\Omega)}=||(u-\tilde{\phi})''||_{L^2(\Omega)}=||(\tilde{\phi}+\epsilon P-\tilde{\phi})''||_{L^2(\Omega)}=||P''||_{L^2(\Omega)}$$
		Et ainsi, en prenant $\hat{\phi}=\tilde{\phi}+m$ on obtient le même résultat.
	\end{Rem}
\end{enumerate}

\paragraph{Résultats numériques \\} 

On se place sur le carré $[0,1]^2$. On considère ici la solution analytique suivante
$$u_{ex}(x,y) = S\times\sin(2\pi fx + p)\times\sin(2\pi fy + p)$$ 

et $P$ la perturbation définie par
$$P(x,y)=S\times\sin(2\pi f_px + p_p)\times\sin(2\pi f_py + p_p)$$

avec $p_p=0$ pour que $P=0$ sur $\Gamma$ (et donc $u_p=u_{ex}$ sur $\Gamma$). 

On cherche alors principalement à comparer les erreurs en norme $L^2$ obtenus avec les problèmes \ref{pbc1r} et \ref{pbc2}.

On prendra $S=0.5$ et $p=0$ (c'est-à-dire $g=0$). On fera varier $\epsilon$, $f$ et $f_p$. 

Voici les résultats obtenus :

\begin{minipage}{\linewidth}
	\begin{figure}[H]
		\centering
		\includegraphics[width=0.45\linewidth]{resultats.png}
		\caption{Résultats FEM pour nb\_vert=64}
	\end{figure}
\end{minipage}

Il semblerait ici que les résultats obtenus pour les problèmes \ref{pbc1r} avec $m=1000$ (avant-dernière colonne) et \ref{pbc2} (dernière colonne) soient très proches.

\paragraph{Explication \\}

On cherche ici à comprendre pourquoi on obtient des résultats aussi proches avec les 2 méthodes.

\subparagraph*{Méthode 1 \\}

On cherche à résoudre le problème
\begin{equation}
	\left\{\begin{aligned}
		&-\Delta (\hat{\phi}C)=f \quad &&\Omega \\
		&\hat{u}=g+m \quad &&\Gamma
	\end{aligned}\right. \tag{$\mathcal{C}_1^\mathcal{R}$}
\end{equation}
avec $\hat{u}=\hat{\phi}C+m$ où $\hat{\phi}=\tilde{\phi}+m$ ($m$ une constante).

La décomposition de $\hat{u_h}$ sur la base $(\varphi_1,\dots,\varphi_{N_h})$ de $V_h$ s'écrit pour ce problème

\begin{equation}
	\hat{u_h}=C_h\hat{\phi}=\left(\sum_{i=1}^{N_h}C_i\varphi_i\right)\hat{\phi}(x) \label{decomp_pbc1}
\end{equation}

Or 
\begin{equation}
	C_i=\frac{u(x_i)+m}{\hat{\phi}(x_i)}=\frac{u(x_i)+m}{\tilde{\phi}(x_i)+m} \label{C_i}
\end{equation}
avec
\begin{equation}
	u(x_i)=\tilde{\phi}(x_i)+\epsilon P(x_i) \label{u_x_i}
\end{equation}
et
\begin{equation}
	\tilde{\phi}(x)=\tilde{\phi}(x_i)+(x-x_i)\tilde{\phi}'(x_i) \label{phi_tilde_x}
\end{equation}
De plus
\begin{equation}
	\sum_{i=1}^{N_h}\varphi_i=1 \label{sum_base}
\end{equation}
Avec les 4 relations précédentes, on peut développer \ref{decomp_pbc1} :
\begin{align*}
	\hat{u_h}&=\left(\sum_{i=1}^{N_h}C_i\varphi_i\right)\hat{\phi}(x) \\
	&=\left(\sum_{i=1}^{N_h}\frac{u(x_i)+m}{\tilde{\phi}(x_i)+m}\varphi_i\right)\hat{\phi}(x) \quad \text{par \ref{C_i}} \\
	&=\left(\sum_{i=1}^{N_h}\frac{\tilde{\phi}(x_i)+m+\epsilon P(x_i)}{\tilde{\phi}(x_i)+m}\varphi_i\right)\hat{\phi}(x) \quad \text{par \ref{u_x_i}} \\
	&=\sum_{i=1}^{N_h}\left(1+\epsilon\frac{ P(x_i)}{\tilde{\phi}(x_i)+m}\right)\varphi_i\hat{\phi}(x) \\
	&=\left(\sum_{i=1}^{N_h}\varphi_i\right)\hat{\phi}(x)+\epsilon\sum_{i=1}^{N_h}P(x_i)\frac{\hat{\phi}(x)}{\tilde{\phi}(x_i)+m}\varphi_i \\
	&=\hat{\phi}(x)+\epsilon\sum_{i=1}^{N_h}P(x_i)\frac{\tilde{\phi}(x_i)+m+(x-x_i)\tilde{\phi}'(x_i)}{\tilde{\phi}(x_i)+m}\varphi_i \quad \text{par \ref{phi_tilde_x} et \ref{sum_base}} \\
	&=\hat{\phi}(x)+\epsilon\sum_{i=1}^{N_h}P(x_i)\left(1+\frac{(x-x_i)\tilde{\phi}'(x_i)}{\tilde{\phi}(x_i)+m}\right)\varphi_i \\
	&=\tilde{\phi}(x)+m+\epsilon\sum_{i=1}^{N_h}P(x_i)\left(1+\frac{(x-x_i)\tilde{\phi}'(x_i)}{\tilde{\phi}(x_i)+m}\right)\varphi_i \\
\end{align*}

Ainsi
$$u_h=\hat{u_h}-m=\tilde{\phi}(x)+\epsilon\sum_{i=1}^{N_h}P(x_i)\left(1+\frac{(x-x_i)\tilde{\phi}'(x_i)}{\tilde{\phi}(x_i)+m}\right)\varphi_i$$
et finalement
\begin{equation}
	u_h\xrightarrow[m\to\infty]{} \tilde{\phi}(x)+\epsilon\sum_{i=1}^{N_h}P(x_i)\varphi_i \label{result1}
\end{equation}

\begin{Rem}
	Pour le problème \ref{pbc1}, (équivalent au problème \ref{pbc1r} avec $m=0$), on a 
	$$u_h=\tilde{\phi}(x)+\epsilon\sum_{i=1}^{N_h}P(x_i)\left(1+\frac{(x-x_i)\tilde{\phi}'(x_i)}{\tilde{\phi}(x_i)}\right)\varphi_i$$
\end{Rem}

\subparagraph*{Méthode 2 \\}

On cherche à résoudre le problème
\begin{equation}
	\left\{\begin{aligned}
		&-\Delta C=\tilde{f} \quad &&\Omega \\
		&C=0 \quad &&\Gamma
	\end{aligned}\right. \tag{$\mathcal{C}_2$}
\end{equation}
avec $\tilde{u}=\tilde{\phi}+C$ et $\tilde{f}=f+\Delta\tilde{\phi}$.

La décomposition de $u_h$ sur la base $(\varphi_1,\dots,\varphi_{N_h})$ de $V_h$ s'écrit pour ce problème

\begin{equation}
	u_h=C_h+\tilde{\phi}=\left(\sum_{i=1}^{N_h}C_i\varphi_i\right)+\tilde{\phi}(x) \label{decomp_pbc2}
\end{equation}

Or 
\begin{equation}
	C_i=u(x_i)-\tilde{\phi}(x_i) \label{C_i_2}
\end{equation}
avec
\begin{equation}
	u(x_i)=\tilde{\phi}(x_i)+\epsilon P(x_i) \label{u_x_i_2}
\end{equation}
Avec les 2 relations précédentes, on peut développer \ref{decomp_pbc2} :
\begin{align*}
	u_h&=\tilde{\phi}(x)+\sum_{i=1}^{N_h}C_i\varphi_i \\
	&=\tilde{\phi}(x)+\sum_{i=1}^{N_h}(u(x_i)-\tilde{\phi}(x_i))\varphi_i \quad \text{par \ref{C_i_2}} \\
	&=\tilde{\phi}(x)+\sum_{i=1}^{N_h}(\tilde{\phi}(x_i)+\epsilon P(x_i)-\tilde{\phi}(x_i))\varphi_i \quad \text{par \ref{u_x_i_2}} \\
	u_h&=\tilde{\phi}(x)+\epsilon\sum_{i=1}^{N_h} P(x_i)\varphi_i \numberthis \label{result2}
\end{align*}

Ainsi par \ref{result1} et \ref{result2}, il semblerait que pour le problème \ref{pb1}, les 2 méthodes proposées soit équivalentes (en prenant $m$ grand).


\setcounter{equation}{0}

\subsubsection{Correction avec $\phi$-FEM}

\begin{Rem}
	Le rehaussement avec $\phi$-FEM n'étant pas encore fonctionnel, on comparera seulement la seconde méthode avec la méthode classique (pour $m=0$). On testera par la suite d'imposer les conditions au bord par la méthode duale et finalement on pourra comparer le rehaussement avec la nouvelle méthode de correction.
\end{Rem}


\paragraph{Présentation des 2 méthodes}

\begin{enumerate}[label=\textbullet]
	\item \textbf{Méthode 1 : }
	On souhaite résoudre le problème suivant
	\begin{equation}
		\label{pbc1_phifem}
		\left\{\begin{aligned}
			&-\Delta (\tilde{\phi}C)=f \quad &&\Omega \\
			&C=1 \quad &&\Gamma
		\end{aligned}\right. \tag{$\mathcal{C}_1$}
	\end{equation}
	
	avec $\tilde{u}=\tilde{\phi}C$.
	
	\item \textbf{Méthode 2 : } On souhaite résoudre le problème suivant
	\begin{equation}
		\label{pbc2_phifem}
		\left\{\begin{aligned}
			&-\Delta(\phi C)=\tilde{f} \quad &&\Omega \\
			&\tilde{C}=0 \quad &&\Gamma
		\end{aligned}\right. \tag{$\mathcal{C}_2$}
	\end{equation}
	
	avec $\tilde{u}=\tilde{\phi}+\tilde{C}$ où $\tilde{C}=\phi C$ et $\tilde{f}=f+\Delta\tilde{\phi}$.
\end{enumerate}

\paragraph{Résultats numériques \\}

Nous allons considérer deux cas tests : le premier sera de considérer comme géométrie un carré (similaire au cas test de FEM) et le second sera de considérer un cercle.

\paragraph*{1er cas test : le carré \\}

On se place sur le carré $[0,1]^2$. On prend alors
$$\phi_c (x,y)=||x-0.5||_\infty-0.5$$
pour construire les ensembles $\mathcal{F}_h^\Gamma$ et $\mathcal{T}_h^\Gamma$ nécessaire à $\phi$-FEM.

On considérera alors le domaine environnant $\mathcal{O}=[-0.5,1.5]^2$ et on prendra la levelset
$$\phi (x,y)=x(1-x)y(1-y)$$

On considère la même solution analytique que celle utilisée dans le cas test de FEM :
$$u_{ex}(x,y) = S\times\sin(2\pi fx + p)\times\sin(2\pi fy + p)$$ 

et $P$ la perturbation définie par
$$P(x,y)=S\times\sin(2\pi f_px + p_p)\times\sin(2\pi f_py + p_p)$$

avec $p_p=0$ pour que $P=0$ sur $\Gamma$ (et donc $u_p=u_{ex}$ sur $\Gamma$). 

On cherche alors principalement à comparer les erreurs en norme $L^2$ obtenus avec les problèmes \ref{pbc1_phifem} et \ref{pbc2_phifem}.

On prendra $S=0.5$ et $p=0$ (c'est-à-dire $g=0$). On fera varier $\epsilon$, $f$ et $f_p$. 

Voici les résultats obtenus :

\begin{minipage}{0.48\linewidth}
	\begin{figure}[H]
		\centering
		\includegraphics[width=0.8\linewidth]{phifem_carre_64.png}
		\caption{Résultats sur le carré (nb\_vert=64)}
	\end{figure}
\end{minipage}
\begin{minipage}{0.48\linewidth}
	\begin{figure}[H]
		\centering
		\includegraphics[width=0.8\linewidth]{phifem_carre_128.png}
		\caption{Résultats sur le carré (nb\_vert=128)}
	\end{figure}
\end{minipage}

Il semblerait que les résultats obtenus pour le problème \ref{pbc2_phifem} (colonne "Corr v2") soient meilleurs que ceux obtenus pour le problème \ref{pbc1_phifem} (colonne "Corr"). La colonne "facteur" contient les coefficients "Corr"/"Corr v2".

\paragraph*{2nd cas test : le cercle \\}

On considère $\Omega$ le cercle de rayon $\sqrt{2}/4$ et de centre $(0.5,0.5)$. On prend 
$$\phi(x,y)=-1/8+(x-1/2)^2+(y-1/2)^2$$
On considère le domaine fictif $O=(0,1)^2$.

On considère toujours la solution analytique suivante :
$$u_{ex}(x,y) = S\times\sin(2\pi fx + \varphi)\times\sin(2\pi fy + \varphi)$$ 

On prend dans ce cas la perturbation $P$ définie par
$$P(x,y) = S\times\sin(2\pi fx + \varphi)\times\sin(2\pi fy + \varphi)\times\cos(4\pi((x-0.5)^2+(y-0.5)^2))$$ 
pour que $P=0$ sur $\Gamma$ (et donc $u_p=u_{ex}$ sur $\Gamma$). 

On cherche comme pour le cas test du carré à comparer les erreurs en norme $L^2$ obtenues avec les problèmes \ref{pbc1_phifem} et \ref{pbc2_phifem}.

On prendra $S=0.5$ et $p=0$ (attention ici $g\ne 0$). On fera varier $\epsilon$, $f$ et $f_p$. 

\begin{Rem}
	Ici, on prend 2 fois moins de nœuds que dans le cas test du carré pour avoir des cas comparables (car le domain $\mathcal{O}$ est deux fois plus petit sur la longueur et sur la largeur).
\end{Rem}

Voici les résultats obtenus :

\begin{minipage}{0.48\linewidth}
	\begin{figure}[H]
		\centering
		\includegraphics[width=0.8\linewidth]{phifem_cercle_32.png}
		\caption{Résultats sur le cercle (nb\_vert=32)}
	\end{figure}
\end{minipage}
\begin{minipage}{0.48\linewidth}
	\begin{figure}[H]
		\centering
		\includegraphics[width=0.8\linewidth]{phifem_cercle_64.png}
		\caption{Résultats sur le cercle (nb\_vert=64)}
	\end{figure}
\end{minipage}

Il semblerait que les résultats obtenus pour le problème \ref{pbc2_phifem} (colonne "Corr v2") soient meilleurs que ceux obtenus pour le problème \ref{pbc1_phifem} (colonne "Corr"). La colonne "facteur" contient les coefficients "Corr"/"Corr v2". On obtient alors le même type de résultat que pour le cas test du carré.

\paragraph{Explication}

On cherche ici à expliciter la forme de la solution dans le problème \ref{pbc2_phifem}.

La décomposition de $u_h$ sur la base $(\varphi_1,\dots,\varphi_{N_h})$ de $V_h$ s'écrit pour ce problème

\begin{equation}
	u_h=\phi(x)C_h+\tilde{\phi}(x) \label{decomp_u_phifem}
\end{equation}
avec
\begin{equation}
	C_h=\sum_{i=1}^{N_h}C_i\varphi_i \label{decomp_C_phifem}
\end{equation}

Or 
\begin{equation}
	C_i=\frac{u(x_i)-\tilde{\phi}(x_i)}{\phi(x_i)} \label{C_i_phifem}
\end{equation}
avec
\begin{equation}
	u(x_i)=\tilde{\phi}(x_i)+\epsilon P(x_i) \label{u_x_i_phifem}
\end{equation}
et
\begin{equation}
	\phi(x)=\phi(x_i)+(x-x_i)\phi'(x_i) \label{phi_x_phifem}
\end{equation}

Les relations \ref{decomp_u_phifem} et \ref{decomp_C_phifem} deviennent alors :
\begin{align*}
	u_h&=\tilde{\phi}(x)+\phi(x)C_h \\
	&=\tilde{\phi}(x)+\phi(x)\sum_{i=1}^{N_h}C_i\varphi_i \\
	&=\tilde{\phi}(x)+\phi(x)\sum_{i=1}^{N_h}\frac{u(x_i)-\tilde{\phi}(x_i)}{\phi(x_i)}\varphi_i \quad \text{par \ref{C_i_phifem}} \\
	&=\tilde{\phi}(x)+\phi(x)\sum_{i=1}^{N_h}\frac{\tilde{\phi}(x_i)+\epsilon P(x_i)-\tilde{\phi}(x_i)}{\phi(x_i)}\varphi_i \quad \text{par \ref{u_x_i_phifem}} \\
	&=\tilde{\phi}(x)+\epsilon\sum_{i=1}^{N_h}P(x_i)\frac{\phi(x)}{\phi(x_i)}\varphi_i \\
	u_h&=\tilde{\phi}(x)+\epsilon\sum_{i=1}^{N_h}P(x_i)\left(1+\frac{(x-x_i)\phi'(x_i)}{\phi(x_i)}\right)\varphi_i \quad \text{par \ref{phi_x_phifem}} \numberthis \label{result_phifem}
\end{align*}

\conclusion{On s'est rendu compte en fin de semaine que les résultats obtenus avaient été obtenus en prenant $\tilde{\phi}$ de degré 10 (et pas de degré 2 comme on pourrait avoir avec le FNO). Les résultats obtenus avec $\tilde{\phi}$ de degré 2 seront à la semaine suivante, ainsi que les résultats obtenus avec le FNO.}